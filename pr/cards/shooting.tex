\documentclass{article}

\usepackage[utf8]{inputenc}
\usepackage[english,russian]{babel}
\usepackage{amsfonts}
\usepackage{amsmath}
\usepackage{eulervm}

\providecommand{\RR}{\mathbb{R}}

\title{Задание 3}

\begin{document}
\section{Задача}
В мишень, представляющую собой прямоугольник \[ \Omega = [0,3]\times [0,2] \]
стреляют из пистолета.
Известно, что отклонение пули от точки, на которую нацелен пистолет, произвольно,
но не превышает \( 0.1 \) по любому из направлений,
параллельным сторонам прямоугольника.
Точнее, отклонение равномерно распределено в квадрате \[ Q = [0,0.1]\times [0,0.1] \]
Стрелок целится в произвольную точку мишени (снова равномерно).
С какой вероятностью стрелок попадёт в мишень?
\section{Решение 1}
\subsection{Формальная постановка}
Даны две случайные величины:
\( X \) -- точка прицеливания, и \( Y \) -- отклонение пули.
Эти величины распределены равномерно:

\[ X: \Omega\to\RR^2 \]
\[ f_X(x) =
\left\{\begin{aligned}
    & \frac{1}{0.04} = 25 & \text{if } x\in\Omega \\
    & 0 & \text{otherwise}
\end{aligned}\right. \]
Короче: \( X\in\mathrm{Uniff}(\Omega) \)

\[ Y: Q\to \RR^2 \]
\[ Y\in\mathrm{Uniff}(Q) \]

Рассмотрим случайную величину
\[ Z: \Omega\times Q \to \RR^2 \]
\[ Z = X + Y \]

Ясно, что исходную задачу можно сформулировать так:
найти
\[ p = \mathbb{P}\left\{ \omega; Z(\omega)\in\Omega \right\} \]


Если \( \rho: \RR^2\to\RR \) --- плотность распределения вероятностей r.v. \( Z \), то
\[ p = \iint_{\RR^2} \rho = \iint_\Omega \rho \]

Заметим, что \( f_Z \) равна нулю вне \( \Omega \) и симметрична, поэтому

\begin{align*}
    p = \iint_\Omega \rho = &\iint_{[0.1,2.9]\times [0.1,1.9]} \rho_I  \\
    & + 4\iint_{[0,0.1]^2} \rho_{II} \\
    & + 2\iint_{[0.1,2.9]\times [0,0.1]} \rho_{2.8} \\
    & + 2\iint_{[0.1,1.9]\times [0,0.1]} \rho_{1.8}
\end{align*}
Где \( \rho_I \) -- плотность внутреннего сегмента,
\( \rho_{II} \) -- плотность в левом нижнем углу,
\( \rho_{a} \) -- площадь нижнего бокового сегмента длины \( a \in \{1.7, 2.7\} \).


Чтобы найти плотность \( \rho \) рассмотрим случайную величину \( (Z=X+Y, Z^\star = Y ) \).
\[\left\{\begin{aligned}
    & X = Z - Z^\star \\
    & Y = Z^\star
\end{aligned}\right.\]
\[ J = \begin{vmatrix}
    1 & -1 \\
    0 & 1
\end{vmatrix} = 1 \]

Ясно, что
\[
    f_{(Z,Z^\star)}(z,z^\star) = f_X(z-z^\star) f_Y(z^\star)
\]
\[
    f_Z(z) = \iint_{\RR^2} f_X(z - z^\star) f_Y(z^\star) \left|J\right| \mathrm{d} z^\star
    = \iint_Q f_X(z - z^\star) f_Y(z^\star) \mathrm{d} z^\star
\]

\subsection{Внутренность}
Рассмотрим отдельно \( z\in [0.1, 2.8]\times [0.1, 1.8] \).
В этой области \( z - z^\star \in \Omega \) и
\[
    f_Z(z) = \rho_I(z) = \iint_Q \frac{25}{6} = 1/6
\]
\[ \iint \rho_I = \frac{2.8\cdot 1.8}{6} = \frac{21}{25} \]

\subsection{Углы}
Рассмотрим \( z\in [0, 0.1]^2 \), \( z-z^\star \in [-0.1,0.2]^2 \)
\[
    f_Z(z) = \rho_{II}(z) = \int_{-z_1}^{0.1} \int_{-z_2}^{0.1} \frac{25}{6} = \frac{25}{6}(0.1+z_1)(0.1+z_2)
\]
\[
    \iint_{[0,0.1]^2} \rho_{II} = \frac{3}{3200}
    \]
\subsection{Боковые сегменты}
Пусть \( z\in [0.1, a]\times [0, 0.1] \)
\[
    f_Z(z) = \rho_{a}(z) = \int_{-0.1}^{0.1} \int_{-z_2}^{0.1} \frac{25}{6} = \frac{5}{6}(0.1 + z_2)
\]
\[
    \int_{0.1}^{0.1+a}\int_{0}^{0.1} \rho_a \mathrm{d}z_2 \mathrm{d} z_1 = \frac{a}{80}
    \]

\subsection{Общая вероятность}
\[
    p = \frac{21}{25} + 4\frac{3}{3200} + 2\frac{2.8+1.8}{80} = 767/800 \approx 0.95875
    \]
\end{document}
