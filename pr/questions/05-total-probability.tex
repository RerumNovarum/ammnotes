\subsection{Закон полной вероятности. Формула
Байеса}\label{ux437ux430ux43aux43eux43d-ux43fux43eux43bux43dux43eux439-ux432ux435ux440ux43eux44fux442ux43dux43eux441ux442ux438.-ux444ux43eux440ux43cux443ux43bux430-ux431ux430ux439ux435ux441ux430}

\subsubsection{Формула полной
вероятности}\label{ux444ux43eux440ux43cux443ux43bux430-ux43fux43eux43bux43dux43eux439-ux432ux435ux440ux43eux44fux442ux43dux43eux441ux442ux438}

Пусть \(\{ A_j, j\in J\}\) --- \emph{разбиение} пространства \(\Omega\)
(\emph{полная группа несовместных событий}). Тогда
\(\forall B \quad B = \sum_j B A_j\), а потому
\(\forall B \quad \pr B = \sum_j \pr B A_j\). Но
\(\pr B A_j = \frac{\pr B A_j}{\pr A_j} \pr A_j = \pr A_j \pr_{A_j} B\).
\[ \pr B = \sum_j \pr_{A_j} B \pr A_j \]

\subsubsection{Формула
Байеса}\label{ux444ux43eux440ux43cux443ux43bux430-ux431ux430ux439ux435ux441ux430}

Пусть даны два события \(A, B\), такие что \(\pr A, \pr B > 0\). Тогда,
как установлено выше:

\[\begin{aligned}
& \pr AB = \frac{\pr AB}{\pr B} \pr B = \pr B \pr_B A \\
& \pr AB = \frac{\pr AB}{\pr A} \pr A = \pr A \pr_A B \\
& \pr B \pr_B A = \pr A \pr_A B \\
& \pr_B A = \frac{\pr A \pr_A B}{\pr B}
\end{aligned}\]

Последняя формула носит имя формулы Байеса:
\[\pr_B A = \frac{\pr A \pr_A B}{\pr B}\]

Более того, если \(A_1, A_2, \ldots, A_n\) --- разбиение \(\Omega\), то

\[\begin{aligned}
& \pr_{A_j} B = \frac{\pr B \pr_B A_j}{\pr A_j} \\
& \pr B   = \sum_j \pr A_j \pr_{A_j} B \\
& \pr_B A_k = \frac{\pr A_k B}{\pr B} = \frac{\pr B \pr_{A_k} B}{\pr B} = \frac{\pr B \pr_{A_k} B}{\sum_j \pr A_j \pr_{A_j} B}
\end{aligned}\]

Это --- расширенная формула Байеса:
\[\pr_B A_k = \frac{\pr B \pr_{A_k} B}{\sum_j \pr A_j \pr_{A_j} B}\]

События \(A_1, A_2, \ldots, A_n\), образующие разбиение \(\Omega\)
называют \emph{гипотезами}, вероятность \(\pr A_i\) называют
\emph{априорной} вероятностью гипотезы, а условные вероятности
\(\pr_B A_i\) называют \emph{апостериорными} вероятностями гипотез
\(A_i\) при наступлении события \(B\)
