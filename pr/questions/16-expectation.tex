\subsection{Математическое ожидание случайной величины и функции
случайной
величины}\label{ux43cux430ux442ux435ux43cux430ux442ux438ux447ux435ux441ux43aux43eux435-ux43eux436ux438ux434ux430ux43dux438ux435-ux441ux43bux443ux447ux430ux439ux43dux43eux439-ux432ux435ux43bux438ux447ux438ux43dux44b-ux438-ux444ux443ux43dux43aux446ux438ux438-ux441ux43bux443ux447ux430ux439ux43dux43eux439-ux432ux435ux43bux438ux447ux438ux43dux44b}

\subsubsection{Def. (Интеграл простой неотрицательной
функции)}\label{def.-ux438ux43dux442ux435ux433ux440ux430ux43b-ux43fux440ux43eux441ux442ux43eux439-ux43dux435ux43eux442ux440ux438ux446ux430ux442ux435ux43bux44cux43dux43eux439-ux444ux443ux43dux43aux446ux438ux438}

\emph{Интеграл простой неотрицательной функции
\(X = \sum_{j=1}^m x_j I_{A_j}\) на \(\Omega\)} есть величина
\[ \int_\Omega X\dmu = \sum_{j=1}^m x_j \mu A_j\]

\subsubsection{Def. (Интеграл неотрицательной измеримой
функции)}\label{def.-ux438ux43dux442ux435ux433ux440ux430ux43b-ux43dux435ux43eux442ux440ux438ux446ux430ux442ux435ux43bux44cux43dux43eux439-ux438ux437ux43cux435ux440ux438ux43cux43eux439-ux444ux443ux43dux43aux446ux438ux438}

Неотрицательная измеримая функция представима в виде \(X = \lim_n X_n\)
предела неубывающей сходящейся последовательности простых функций:
\(X_n \uparrow X\). Её интеграл по \(\Omega\) определим как
\[ \int_\Omega X\dmu = \lim_n \int_\Omega X_n \dmu \]

\subsubsection{Def. (Интеграл измеримой
функции)}\label{def.-ux438ux43dux442ux435ux433ux440ux430ux43b-ux438ux437ux43cux435ux440ux438ux43cux43eux439-ux444ux443ux43dux43aux446ux438ux438}

Всякая измеримая функция представима в виде \(X = X^+ - X^-\). Интеграл
по \(\Omega\) такой функции определим как
\[ \int_\Omega X \dmu = \int_\Omega X^+ \dmu - \int_\Omega X^- \dmu \]

\subsubsection{Def. (Интеграл по
множеству)}\label{def.-ux438ux43dux442ux435ux433ux440ux430ux43b-ux43fux43e-ux43cux43dux43eux436ux435ux441ux442ux432ux443}

Интеграл измеримой функции \(X\) по измеримому множеству
\(A \in \SigmaField\) определим как
\[ \int_A X\dmu = \int_\Omega X I_A \dmu \]

\subsubsection{Def. (Интегрируемая
функция)}\label{def.-ux438ux43dux442ux435ux433ux440ux438ux440ux443ux435ux43cux430ux44f-ux444ux443ux43dux43aux446ux438ux44f}

Функция \(X\) называется \emph{интегрируемой по множеству \(A\)}, если
\(\int_A X\dmu\) существует и конечен.

\subsubsection{Свойства
интеграла}\label{ux441ux432ux43eux439ux441ux442ux432ux430-ux438ux43dux442ux435ux433ux440ux430ux43bux430}

Интеграл по множеству является линейным функционалом из пространства
интегрируемых функций:
\({\int_A \cdot : X \mapsto \int_\Omega X \dmu}\).

Пусть \(X = \sum_{j=1}^n x_j I_{A_j}\), \(Y = \sum_{k=1}^m y_k I_{B_k}\)
И пусть существуют и конечны \(\int_A X\), \(\int_A Y\) и
\(\int_A X + \int_A Y\).

Тогда

\(\int_A \alpha X \dmu = \sum_j \alpha x_j \pr A_j = \alpha \sum_j x_j \pr A_j = \alpha \int_A X\dmu\).

\(\int_A X+Y \dmu = \sum_{j,k} (x_j + y_k) I_{A_j} I_{B_k} = \sum_j x_j I_{A_j} + \sum_k y_k I_{B_k} = \int_A X\dmu + \int_A Y\dmu\)

Общий случай доказывается предельным переходом. Нужно только уметь
доказывать, что \(\forall X \exists X_n \quad {X_n \uparrow X}\) (как?).
Тогда для неотрицательной измеримой функции монотонная
последовательность \(\int_A X_n\dmu\) имеет предел (конечный или
бесконечный) и нужно доказать единственность (как?).

\subsubsection{Def. (Математическое
ожидание)}\label{def.-ux43cux430ux442ux435ux43cux430ux442ux438ux447ux435ux441ux43aux43eux435-ux43eux436ux438ux434ux430ux43dux438ux435}

\emph{Математическим ожиданием} случайной величины
\(X : \Omega\mapsto\RR\) называется интеграл \(\int_\Omega X \dd\pr\) по
\(\Omega\).
