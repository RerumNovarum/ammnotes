\subsection{Плотность вероятности функции от случайной
величины}\label{ux43fux43bux43eux442ux43dux43eux441ux442ux44c-ux432ux435ux440ux43eux44fux442ux43dux43eux441ux442ux438-ux444ux443ux43dux43aux446ux438ux438-ux43eux442-ux441ux43bux443ux447ux430ux439ux43dux43eux439-ux432ux435ux43bux438ux447ux438ux43dux44b}

\subsubsection{Def. (Борелевская
функция)}\label{def.-ux431ux43eux440ux435ux43bux435ux432ux441ux43aux430ux44f-ux444ux443ux43dux43aux446ux438ux44f}

Борелевской называется измеримая функция вида
\(g : (\RR^n, \BB^n)\mapsto (\RR^m,\BB^m)\). То есть функция,
относительно которой прообразами борелевских множеств являются
борелевские множества.

\subsubsection{Def. (Борелевская фунция от случайной
величины)}\label{def.-ux431ux43eux440ux435ux43bux435ux432ux441ux43aux430ux44f-ux444ux443ux43dux446ux438ux44f-ux43eux442-ux441ux43bux443ux447ux430ux439ux43dux43eux439-ux432ux435ux43bux438ux447ux438ux43dux44b}

Пусть \(X = (X_1, \ldots, X_n)\) --- r.v. на \((\Omega, \FF, \pr)\).
\({\Psi_j, j=\overline{1,k}}\) --- борелевские функции
\({\Psi_j : X(\Omega)\mapsto\RR}\).
\({\Psi = (\Psi_1, \ldots, \Psi_k)}\),
\({\Psi : X(\Omega)\mapsto\Gamma, \quad \Gamma\subset\RR^k}\) ---
отображение из области \(X(\Omega)\) значений r.v. \(X\) в область
\(k\)-мерного вещественного пространства.

\(Y = (Y_1, \ldots, Y_k), \quad Y_j = \Psi_j(X)\) --- случайная величина
\(Y: \Omega\mapsto\Gamma\)

\subsubsection{\texorpdfstring{Задача: найти плотность \(f_Y\) по данной
плотности
\(f_X\)}{Задача: найти плотность f\_Y по данной плотности f\_X}}\label{ux437ux430ux434ux430ux447ux430-ux43dux430ux439ux442ux438-ux43fux43bux43eux442ux43dux43eux441ux442ux44c-fux5fy-ux43fux43e-ux434ux430ux43dux43dux43eux439-ux43fux43bux43eux442ux43dux43eux441ux442ux438-fux5fx}

\begin{enumerate}
\def\labelenumi{\arabic{enumi}.}
\item
  \(f_Y (y_1,\ldots,y_k) = \frac{\partial^k F(y_1, \ldots, y_k)}{\partial y_1 \ldots \partial y_k}\)
\item
  \(A = \Psi^{-1}(B)\)
  \(\pr\{ Y(\omega)\in B\} = \pr\{ X(\omega)\in A\}\)
  \(\idotsint_B f_Y(y_1,\ldots,y_k) \dd y_1\ldots \dd y_k = \idotsint_A f_X (x_1,\ldots,x_n) \dd x_1 \ldots \dd x_n\)
\item
  В частности, пусть \(k=n\).

  \(\Psi\in C^{\infty} (X(\Omega), \Gamma), \Psi\uparrow,\)
  \(x = \Psi^{-1}(y)\)

  \[J(y) = \frac{\partial\Psi^{-1}}{\partial y} =
    \begin{vmatrix}
    &\frac{\partial\Psi_1^{-1} (y)}{\partial y_1} & \ldots & \frac{ \partial \Psi_1^{-1}(y)}{\partial y_n} \\
    &\vdots & \ddots & \vdots \\
    &\frac{\partial\Psi_n^{-1} (y)}{\partial y_1} & \ldots & \frac{ \partial \Psi_n^{-1}(y)}{\partial y_n}
    \end{vmatrix}
  \]

  \[f_Y (y) = f_X (\Psi(y)) \left|{\frac{\partial\Psi^{-1}}{\partial y}}\right|\]

  Пример:

  \(X\sim \Gaussian(0,1)\).

  \(f_X = \frac{1}{\sqrt{2\pi}}\exp{-\frac{x^2}{2}}\).

  \(Y = aX + b, a\neq 0\). \(X = \frac{Y-b}{a}\).

  \(J = \frac{1}{a}\)

  \(f_Y(y) = f_X(\Psi^{-1}(y))|J(y)| = \frac{1}{|a|\sqrt{2\pi}} \exp{-\frac{(y-b)^2}{2a^2}}\)

  То есть \(Y\sim \Gaussian(b, |a|)\).
\item
  Пусть \(k < n\). ` \(X = (X_1, \ldots, X_n)\),
  \(Y = (\Psi_1(X), \ldots, \Psi_k(X))\)

  Введём

  \(Y^\star = (Y_1, \ldots, Y_n) := (\Psi_1(X), \ldots, \Psi_k(X), X_{k+1}, \ldots, X_{n})\)

  \({y^\star = (y_1, \ldots, y_n)}\in\RR^n, {y = (y_1,\ldots,y_k) \in\Gamma}\)

  \[f_{Y^\star}(y^\star) = f_X(\Psi_1^{-1}(y_1,\ldots,y_k), \ldots, \Psi_k^{-1}(y_1,\ldots,y_k), y_{k+1}, \ldots, y_{n})\]

  \[\begin{aligned}
    & f_Y(y) &&=
      \idotsint_{-\infty}^{+\infty}
      f_{Y^\star} (y) \dd y_{k+1} \ldots \dd y_{n} = \\
    & &&= \idotsint_{-\infty}^{+\infty}
        f_X(\Psi_1^{-1}(y_1,\ldots,y_k), \ldots, \Psi_k^{-1}(y_1,\ldots,y_k), y_{k+1}, \ldots, y_{n})
        |J(y)|
        \dd y_{k+1} \ldots \dd y_{n}
  \end{aligned}\]
\end{enumerate}
