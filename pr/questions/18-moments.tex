\subsection{Моменты случайной величины. Коэффициент корреляции.
Коэффициенты асимметрии и эксцесса. Квантиль
распределения}\label{ux43cux43eux43cux435ux43dux442ux44b-ux441ux43bux443ux447ux430ux439ux43dux43eux439-ux432ux435ux43bux438ux447ux438ux43dux44b.-ux43aux43eux44dux444ux444ux438ux446ux438ux435ux43dux442-ux43aux43eux440ux440ux435ux43bux44fux446ux438ux438.-ux43aux43eux44dux444ux444ux438ux446ux438ux435ux43dux442ux44b-ux430ux441ux438ux43cux43cux435ux442ux440ux438ux438-ux438-ux44dux43aux441ux446ux435ux441ux441ux430.-ux43aux432ux430ux43dux442ux438ux43bux44c-ux440ux430ux441ux43fux440ux435ux434ux435ux43bux435ux43dux438ux44f}

Математические ожидание степеней случайной величины назыаются её
моментами

\subsubsection{Def. (k-й момент случайной
величины)}\label{def.-k-ux439-ux43cux43eux43cux435ux43dux442-ux441ux43bux443ux447ux430ux439ux43dux43eux439-ux432ux435ux43bux438ux447ux438ux43dux44b}

k-й момент случайной величины \(X\) есть \(\E X^k, k>0\)

\subsubsection{Def. (k-й абсолютный момент случайной
величины)}\label{def.-k-ux439-ux430ux431ux441ux43eux43bux44eux442ux43dux44bux439-ux43cux43eux43cux435ux43dux442-ux441ux43bux443ux447ux430ux439ux43dux43eux439-ux432ux435ux43bux438ux447ux438ux43dux44b}

k-й абсолютный момент r.v. \(X\) есть \(\E |X|^k, k>0\)

\(k\)-й момент случаной величины может не существовать, в то время как
\(k\)-й абсолютный момент всегда существует, быть может бесконечный. А
так как интегрируемость эквивалентна абсолютной интегрируемости, то из
конечности \(k\)-го абсолютного момента \(\E |X|^k\) вытекает
существование и конечность \(k\)-го момента \(\E X^k\).

Кроме того из неравенства
\(\forall k\prime\leq k \quad ( |X|^{k\prime} \leq 1 + |X|^k )\)
вытекает, что конечность \(k\)-го абсолютного момента означает
существование и конечность всех (в т.ч. абсолютных) моментов меньших
порядков.

\subsubsection{\texorpdfstring{Def. (k-й \textbf{центрированный}
момент)}{Def. (k-й центрированный момент)}}\label{def.-k-ux439-ux446ux435ux43dux442ux440ux438ux440ux43eux432ux430ux43dux43dux44bux439-ux43cux43eux43cux435ux43dux442}

k-й центрированный момент r.v. \(X\) есть \(\E (X - \E X)^k\)

Видно, что \(\D X\) есть \(k\)-й центрированный момент второго порядка.

\subsubsection{Def.
(Ковариация)}\label{def.-ux43aux43eux432ux430ux440ux438ux430ux446ux438ux44f}

Пусть \(X, Y\) --- случайные величины и \(\exists\) конечные
\(\E X, \E Y\). Их ковариацией называется величина
\[ \cov(X, Y) = \E( (X - \E X)(Y - \E Y) ) \]

Если \(\cov(X,Y) = 0\), говорят, что \(X\) и \(Y\) не коррелированны.

NB: Если \(X\), \(Y\) независимы, то \(\cov(X,Y)=0\)

Пусть \(0 < \D X, \D Y < \infty\). Тогда величина
\[ \rho(X,Y) = \frac{\cov(X,Y)}{\sqrt{\D X \D Y}} \] называется
коэффициентом коррелляции r.v. X и Y.

Свойства:

\begin{enumerate}
\def\labelenumi{\arabic{enumi}.}

\item
  \(|\rho(X,Y)| \leq 1\)
\item
  \(|\rho(X,Y)| = 1 \implies X,Y\) --- линейно-зависимы
\item
  \(Y=aX+b, a\neq 0 \implies |\rho(X,Y)|=1\)
\item
  \(X,Y\) --- независимы \(\implies \rho(X,Y)=0\)
\end{enumerate}

Пусть \(X = (X_1, \ldots, X_n)\) -- r. вектор, компоненты которого имеют
конечный второй момент. Матрицей ковариаций r. вектора \(X\) называют
матрицу \[ R = (\cov(X_i, X_j))_{n\times n} \] Видно, что \(R\) ---
симметричная, кроме того она неотрицательно определена:
\[\forall \lambda_i\in\RR\quad \sum_{i,j}^n R_{ij} \lambda_i \lambda_j = \E \left[{\sum_{i=1}^n (X_i - \E X_i)\lambda_i}\right]^2 \geq 0\]

\subsubsection{Def. (Ковариация векторных случайных
величин)}\label{def.-ux43aux43eux432ux430ux440ux438ux430ux446ux438ux44f-ux432ux435ux43aux442ux43eux440ux43dux44bux445-ux441ux43bux443ux447ux430ux439ux43dux44bux445-ux432ux435ux43bux438ux447ux438ux43d}

Представляет собой корреляционную матрицу. Пусть
\(X = (X_1, \ldots, X_n), Y = (Y_1, \ldots, Y_n)\).

\[\cov(X,Y) = (\cov (X_i, Y_j))_{n\times n} =
\begin{pmatrix}
&\cov(X_1, Y_1) & \cdots & \cov(X_1, Y_n) \\
&\vdots         & \ddots & \vdots         \\
&\cov(X_n, Y_1) & \cdots & \cov(X_n, Y_n)
\end{pmatrix}\]

\begin{center}\rule{0.5\linewidth}{\linethickness}\end{center}

Пусть \({\mu_k = \E (X - \E X)^3}\) --- центральный момент \(k\)-го
порядка, \({\sigma = +\sqrt{\D X}}\) --- стандартное отклонение.

\subsubsection{Def. (Коэффициент
ассиметрии)}\label{def.-ux43aux43eux44dux444ux444ux438ux446ux438ux435ux43dux442-ux430ux441ux441ux438ux43cux435ux442ux440ux438ux438}

Коэффициентом ассиметрии r.v. \(X\) называется величина
\[\gamma_1 = \frac{\mu_3}{\sigma^3} = \frac{ \E (X-\E X)^3 }{ (\D X)^{\frac{1}{3}}}\]

Если распределение симметрично относительно математического ожидания, то
его коэффициент асимметрии равен нулю.

\subsubsection{Def. (Коэффициент
эксцесса)}\label{def.-ux43aux43eux44dux444ux444ux438ux446ux438ux435ux43dux442-ux44dux43aux441ux446ux435ux441ux441ux430}

Коэффициентом эксцесса r.v. \(X\) называется величина
\[\gamma_2 = \frac{\mu_4}{\sigma^4} - 3\]

\subsubsection{Def. (Квантиль
распределения)}\label{def.-ux43aux432ux430ux43dux442ux438ux43bux44c-ux440ux430ux441ux43fux440ux435ux434ux435ux43bux435ux43dux438ux44f}

Квантиль распределения r.v. \(X\) порядка \(p\in [0,1]\) \(:=\) значение
\(x_p\in X(\Omega):\quad F(x_p) = p\).

\subsubsection{Def.
(Медиана)}\label{def.-ux43cux435ux434ux438ux430ux43dux430}

Медианой называется квантиль порядка \(p=\frac{1}{2}\)

\subsubsection{Def. (Мода)}\label{def.-ux43cux43eux434ux430}

Точка \(x\in X(\Omega)\), в которой плотность \(f_X\) принимает
\emph{наибольшее} значение. Если такая точка единственная, rv называется
\emph{унимодальной}, иначе \emph{полимодальной}.
