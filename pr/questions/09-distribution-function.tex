\subsection{Функции распределения. Плотность
вероятности}\label{ux444ux443ux43dux43aux446ux438ux438-ux440ux430ux441ux43fux440ux435ux434ux435ux43bux435ux43dux438ux44f.-ux43fux43bux43eux442ux43dux43eux441ux442ux44c-ux432ux435ux440ux43eux44fux442ux43dux43eux441ux442ux438}

\subsubsection{Def. (Функция
распределения)}\label{def.-ux444ux443ux43dux43aux446ux438ux44f-ux440ux430ux441ux43fux440ux435ux434ux435ux43bux435ux43dux438ux44f}

Под \emph{функцией распределения}, обозначаемой \(F\) с индексами или
без, понимают определённую на \(\RR\) неубывающую непрерывную слева
функцию с значениями в отрезке \([0,1]\)

Очевидно, что величины
\(F(-\infty) := \lim_{x\to -\infty} F(x) = \inf F\),
\(F(+\infty) := \lim_{x\to +\infty} F(x) = \sup F\),
\(F(x) = F(x - 0) := \lim_{x_n\uparrow x} F(x_n) = \sup\limits_{t<x} F(t)\),
\(F(x - 0) := \lim_{x_n\downarrow x} F(x_n) = \inf\limits_{t>x} F(t)\)

существуют и лежат в отрезке \([0,1]\).

\emph{Функция распределения} всегда является функцией распределения
некоторой измеримой функции, определённой на \(|Omega\).

Если вдобавок \((F(-\infty) = 0) \land (F(+\infty) = 1)\), то \(F\)
является функцией распределения некоторой случайной величины.

\subsection{Th.}\label{th.}

Пусть \(F\) --- функция распределения некоторой r.v. Тогда
\(\exists! \pr\), такая что \(\forall a<b\) \[\pr [a,b) = F(b) - F(a)\]

Действительно, пусть \(\FF = \BB\) --- борелева алгебра, а вероятность
множества \(A = \sum_j [a_j,b_j)\) задаётся соотношением
\(\pr A = \sum_j F(b_j) - F(a_j)\).

\subsubsection{Дискретная
мера}\label{ux434ux438ux441ux43aux440ux435ux442ux43dux430ux44f-ux43cux435ux440ux430}

\emph{Дискретной} называют меру, функция распределения которой
\emph{кусочно-постоянна}, меняет свои значения в точках
\(x_1, x_2, \ldots\).
\(\pr\{x_k\} = \Delta F(x_k) = F(x_k+0) - F(x_k-0)\),
\(\sum_k \pr \{x_k\} = 1\). Набор чисел \(p_1, p_2, \ldots\), где
\(p_k = \pr \{x_k\}\) называется \emph{дискретным распределением
вероятностей}, а функция \(F\) распределения называется \emph{дискретной
функцией распределения}.

\subsubsection{Абсолютно-непрерывная
мера}\label{ux430ux431ux441ux43eux43bux44eux442ux43dux43e-ux43dux435ux43fux440ux435ux440ux44bux432ux43dux430ux44f-ux43cux435ux440ux430}

\emph{Абсолютно-непрерывной} называют меру, соотвествующую функции
распределения \(F\), для которой \(\exists f:\RR\mapsto\RR_+\), такая
что \[ F(x) = \int_{-\infty}^x f(t) \dd t\] Где \(\int\) --- интеграл
Римана (а в общем случае --- Лебега).

Функция \(f\) называется \emph{плотностью распределения вероятностей}, а
\(F\) называется \emph{абсолютно-непрерывной функцией распределения}.

Ясно, что любая \(f\), интегрируемая по \(\RR\) и такая что
\(\int_{-\infty}^{+\infty} f(t) \dd t = 1\), определяет некоторую
функцию распределения.

\subsubsection{Сингулярные
меры}\label{ux441ux438ux43dux433ux443ux43bux44fux440ux43dux44bux435-ux43cux435ux440ux44b}

Меры, функции распределения которых непрерывны, но число точек роста
образует несчётное множество.

\subsubsection{Функция распределения случайной
величины}\label{ux444ux443ux43dux43aux446ux438ux44f-ux440ux430ux441ux43fux440ux435ux434ux435ux43bux435ux43dux438ux44f-ux441ux43bux443ux447ux430ux439ux43dux43eux439-ux432ux435ux43bux438ux447ux438ux43dux44b}

Пусть дана функция \(X\). Вероятностная мера \(\pr_X\) на
\((\RR, \BB)\), определяемая формулой
\(\pr_X (B) = \pr\{\omega: X(\omega)\in B\},\quad \forall B\in\BB\)
называется *распределением вероятностей r.v. \(X\) на \((\RR, \BB)\).

Функция \(F_X = \pr \{\omega: X(\omega) \leq x\}, \quad\forall x\in\RR\)
называется \emph{функцией распределения случайной величины \(X\)}.

\paragraph{Дискретная
r.v.}\label{ux434ux438ux441ux43aux440ux435ux442ux43dux430ux44f-r.v.}

Для элементарной r.v. \(X = \sum_j x_j I_{A_j}\) мера \(\pr_X\) может
быть представлена в виде \(\pr_X (B) = \sum_{x_j\in B} \pr\{x_j\}\), где
\(\pr\{x_j\} = \Delta F_X(x_j)\)

\paragraph{Непрерывная
r.v.}\label{ux43dux435ux43fux440ux435ux440ux44bux432ux43dux430ux44f-r.v.}

r.v. \(X\) называется \emph{непрерывной}, если её функция распределения
непрерывна по \(x\).

r.v. \(X\) называется \emph{абсолютно-непрерывной}, если
\(\exists f: \RR\mapsto\RR_+\), называемая плотностью, такая что
\[F_X(x) = \int_{-\infty}^x f_X(t) \dd t, \quad \forall x\in\RR\]
