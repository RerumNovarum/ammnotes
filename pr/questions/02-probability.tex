\subsection{Вероятность и её
свойства}\label{ux432ux435ux440ux43eux44fux442ux43dux43eux441ux442ux44c-ux438-ux435ux451-ux441ux432ux43eux439ux441ux442ux432ux430}

\subsubsection{Def. (Функция
множеств)}\label{def.-ux444ux443ux43dux43aux446ux438ux44f-ux43cux43dux43eux436ux435ux441ux442ux432}

Функцией множеств называется функция вида
\(\phi : S(\Omega)\mapsto\KK\), где \(\KK\) --- числовое поле, быть
может расширенное специальным числом \(\infty\). Далее будут
рассматриваться функции множества вида \(\phi : S(\Omega)\mapsto\RR\)
или \(\phi : S(\Omega)\mapsto [-\infty,+\infty]\)

\subsubsection{Def.
(Аддитивность)}\label{def.-ux430ux434ux434ux438ux442ux438ux432ux43dux43eux441ux442ux44c}

Фукция множеств \(\phi : S(\Omega)\mapsto\KK\) называется аддитивной,
если для любого конечного числа взаимно-непересекающихся множеств
\(A_{1}, \ldots, A_n \in \Omega\) выполняется равенство
\[ \sum_{j=1}^n \phi A_j = \phi \sum_{j=1}^n A_j \]

\subsubsection{\texorpdfstring{Def.
(\(\sigma\)-аддитивность)}{Def. (\textbackslash{}sigma-аддитивность)}}\label{def.-sigma-ux430ux434ux434ux438ux442ux438ux432ux43dux43eux441ux442ux44c}

Функция множеств \(\phi : S(\Omega)\mapsto\KK\) назвыается
\(\sigma\)-аддитивной, если для любого не более чем счётного числа
взаимно-непересекающихся множеств \(A_{j}, j\in J\) выполняется
равенство \[ \sum_{j\in J} \phi A_{j} = \phi \sum_{j\in J} A_{j} \] Где
сумма по \(j\in J\) в левой части равенства понимается в смысле суммы
либо абсолютно сходящегося, либо расходящегося к \(+\infty\) числового
ряда. Сумма в правой части представляет собой либо сумму конечного числа
взаимно-непересекающихся множеств, либо
\(\limsup_n \sum_{j=j_1}^{j_n} A_j\)

\subsubsection{Def. (Мера)}\label{def.-ux43cux435ux440ux430}

\emph{Мерой} называется \(\sigma\)-аддитивная неотрицательная функция
множеств \hbox{$\mu : \SigmaField\mapsto [0,+\infty]$}, определённая на
\(\sigma\)-алгебре \(\SigmaField\)

\subsubsection{Def.
(Вероятность)}\label{def.-ux432ux435ux440ux43eux44fux442ux43dux43eux441ux442ux44c}

Если мера \(\pr\) такова, что \(\pr\Omega = 1\), то она называется
\emph{нормированной} мерой, или \emph{вероятностью}. То есть \(pr\) ---
вероятность, если:

\begin{enumerate}
\def\labelenumi{\arabic{enumi}.}

\item
  \(\pr : \SigmaField\mapsto [0,1]\) (неотрицательность)
\item
  \(\pr A = 1\) (нормированность)
\item
  \(\sum_j \pr A_j = \pr \sum_j A_j\) (\(\sigma\)-аддитивность)
\end{enumerate}

\subsubsection{Def. (Измеримое
пространство)}\label{def.-ux438ux437ux43cux435ux440ux438ux43cux43eux435-ux43fux440ux43eux441ux442ux440ux430ux43dux441ux442ux432ux43e}

Измеримым пространством называется пара \((\Omega, \SigmaField)\),
состоящая из

\begin{itemize}

\item
  пространства \(\Omega\)
\item
  \(\sigma\)-алгебры \(\SigmaField\) измеримых множеств из \(\Omega\)
\end{itemize}

\subsubsection{Def. (Пространство с
вероятностью)}\label{def.-ux43fux440ux43eux441ux442ux440ux430ux43dux441ux442ux432ux43e-ux441-ux432ux435ux440ux43eux44fux442ux43dux43eux441ux442ux44cux44e}

Пространством с вероятностью называется тройка
\((\Omega, \SigmaField, \pr)\), состоящая из

\begin{itemize}

\item
  пространства \(\Omega\)
\item
  \(\sigma\)-алгебры \(\SigmaField\) измеримых множеств (событий) из
  \(\Omega\)
\item
  вероятности \(\pr : \SigmaField\mapsto [0,1]\)
\end{itemize}

\subsubsection{\texorpdfstring{Свойства \(\sigma\)-аддитивных функций,
меры и
вероятности}{Свойства \textbackslash{}sigma-аддитивных функций, меры и вероятности}}\label{ux441ux432ux43eux439ux441ux442ux432ux430-sigma-ux430ux434ux434ux438ux442ux438ux432ux43dux44bux445-ux444ux443ux43dux43aux446ux438ux439-ux43cux435ux440ux44b-ux438-ux432ux435ux440ux43eux44fux442ux43dux43eux441ux442ux438}

Пусть дано измеримое пространство \((\Omega, \SigmaField)\).

\begin{itemize}
\item
  Если некоторая аддитивная функция \(\phi\) множеств конечна хотя бы на
  одном множестве \(A\in \SigmaField\), то \(\phi\emptyset = 0\), так
  как \(\mu A = \mu (A+\emptyset) = \mu A + \mu\emptyset\) Пусть дана
  некоторая мера \(\mu : \SigmaField \mapsto [0,+\infty]\), конечная
  хотя бы на одном множестве \(A\). Тогда:
\item
  \(\mu\emptyset = 0\) так как мера \(\sigma\)-аддитивна, а по условию
  ещё и конечна
\item
  \(\mu A\cup B = \mu A + \mu B - \mu AB\) так как в виду аддитивности и
  очевидных разбиений множеств \(A\) и \(A\cup B\) \[\begin{aligned}
    & \mu A       = \mu AB +  \mu A B^\complement \\
    & \mu A\cup B = \mu B  +  \mu A B^\complement \\
    & \mu A - \mu A\cup B = \mu AB - \mu B \\
    & \mu A + \mu B - \mu AB = \mu A\cup B
  \end{aligned}\]
\item
  \(A_{1}\subset A_{2} \subset\ldots \implies \mu A_{1} \leq \mu A_{2} \leq \ldots\)
  (монотонность) \[\begin{aligned}
    & A_2     &&= A_1 + A_2 A_1^\complement \\
    & \mu A_2 &&= \mu A_1 + \mu A_2 A_1^\complement \\
    & \mu     &&\geq 0 \\
    & \mu A_2 &&\geq \mu A_1 \\
    & A_n     &&= A_1 + A_2 A_1^\complement + A_3 A_2^\complement A_1^\complement + \ldots \\
    & Далее по индукции
  \end{aligned}\]
\item
  \(\mu \cup A_j \leq \sum \mu A_j\) (суб-аддитивность)
  \[\begin{aligned}
    & \cup_{j=1}^\infty A_j = A_1 + A_2 A_1^\complement + \ldots \\
    & \begin{aligned}
      & A_2 A_1^\complement \subset A_2 \\
      & A_3 A_2^\complement A_1^\complement \subset A_3\\
      & \ldots
      \end{aligned} \\
    & \begin{aligned}
        & \mu A_2 \leq \mu (A_2 A_1^\complement) \\
        & \mu A_n \leq \mu (A_{n-1} A_{n-2} \ldots A_1)
      \end{aligned}
    \end{aligned}\]
\item
  \(A \subset B \implies \mu (B - A) = \mu B - \mu A\)

  \(A (B-A) = \emptyset\), т.е. \(A\) и \(B-A\) не пересекаются, а
  значит имеет место аддитивность: \[\begin{aligned}
     & A \cup (B-A) = A + (B-A) \\
     & \mu A\cup (B-A) = \mu A + \mu (B-A) \\
     & \mu (B-A) = \mu A - \mu B
   \end{aligned}\]
\end{itemize}

Если же данная мера --- \emph{вероятность} \(\pr\), то

\begin{enumerate}
\def\labelenumi{\arabic{enumi}.}
\item
  \(\pr A = 1 - \pr A^\complement\)

  \(1 = \pr \Omega = \pr A + \pr A^\complement\)
\end{enumerate}
