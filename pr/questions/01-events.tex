\subsection{События, Классификация событий, Действия над
событиями}\label{ux441ux43eux431ux44bux442ux438ux44f-ux43aux43bux430ux441ux441ux438ux444ux438ux43aux430ux446ux438ux44f-ux441ux43eux431ux44bux442ux438ux439-ux434ux435ux439ux441ux442ux432ux438ux44f-ux43dux430ux434-ux441ux43eux431ux44bux442ux438ux44fux43cux438}

Пусть дано некоторое пространство \(\Omega\). Множество всевозможных
подмножеств \(\Omega\) будем обозначать \(S(\Omega)\)

\subsubsection{\texorpdfstring{Def.
(\(\sigma\)-алгебра)}{Def. (\textbackslash{}sigma-алгебра)}}\label{def.-sigma-ux430ux43bux433ux435ux431ux440ux430}

\(\sigma\)-алгеброй пространства \(\Omega\), называется класс
\(\SigmaField\) множеств \(A\subset\Omega\), такой что

\begin{enumerate}
\def\labelenumi{\arabic{enumi}.}
\item
  для любого не более чем счётного числа множеств
  \(A_{j}\in\SigmaField\) имеет место \[\begin{aligned}
  & \bigcup_{j\in J} A_{j} \in \SigmaField \\
  & \bigcap_{j\in J} A_{j} \in \SigmaField
  \end{aligned}\]
\item
  \(\forall A\in\SigmaField\)\quad

  \(A^\complement \in \SigmaField\)
\end{enumerate}

\paragraph{Следствия из
определения}\label{ux441ux43bux435ux434ux441ux442ux432ux438ux44f-ux438ux437-ux43eux43fux440ux435ux434ux435ux43bux435ux43dux438ux44f}

Полагая, что объединение по пустому множеству индексов есть пустое
множество, а пересечение по пустому множеству индексов есть всё
пространство, получаем

\begin{enumerate}
\def\labelenumi{\arabic{enumi}.}

\item
  \(\emptyset\in\SigmaField\)
\item
  \(\Omega\in\SigmaField\)
\end{enumerate}

\subsubsection{Def.
(Событие)}\label{def.-ux441ux43eux431ux44bux442ux438ux435}

Множества \(A\) из класса \(\SigmaField\) называются \emph{измеримыми
множествами}, или \emph{событиями}

\subsubsection{Def. (Элементарное
событие)}\label{def.-ux44dux43bux435ux43cux435ux43dux442ux430ux440ux43dux43eux435-ux441ux43eux431ux44bux442ux438ux435}

Элементы \(\omega\) пространства \(\Omega\) называются
\emph{элементарными событиями}

\subsubsection{Def. (Невозможное
событие)}\label{def.-ux43dux435ux432ux43eux437ux43cux43eux436ux43dux43eux435-ux441ux43eux431ux44bux442ux438ux435}

Пустое множество \(\emptyset\) называется \emph{невозможным событием}

\subsubsection{Def. (Достоверное
событие)}\label{def.-ux434ux43eux441ux442ux43eux432ux435ux440ux43dux43eux435-ux441ux43eux431ux44bux442ux438ux435}

Всё пространство \(\Omega\) называется \emph{достоверным событием}

\subsubsection{Def. (Несовместные
события)}\label{def.-ux43dux435ux441ux43eux432ux43cux435ux441ux442ux43dux44bux435-ux441ux43eux431ux44bux442ux438ux44f}

Два события \(A\) и \(B\) называются \emph{несовместными}, если
\(AB = \emptyset\)
