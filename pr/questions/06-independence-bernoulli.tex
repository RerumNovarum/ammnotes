\subsection{Независимость испытаний. Схема
Бернулли}\label{ux43dux435ux437ux430ux432ux438ux441ux438ux43cux43eux441ux442ux44c-ux438ux441ux43fux44bux442ux430ux43dux438ux439.-ux441ux445ux435ux43cux430-ux431ux435ux440ux43dux443ux43bux43bux438}

\subsubsection{Def. (Bernoulli trial)}\label{def.-bernoulli-trial}

Эксперимент (случайная величина \(X\)), исходом которого является либо
успех с вероятностью \(p\), либо неудача с вероятностью \((1-p)\),
называется \emph{bernoulli trial}. При этом говорят, что r.v. \(X\)
\emph{распределена по Бернулли с параметром \(p\)} и пишут
\(X\sim\Bern(p)\)

\subsubsection{Пример}\label{ux43fux440ux438ux43cux435ux440}

Любому событию \(A\) соответствует индикатор \(I_A\). Очевидно, он
является \newline
\hbox{r.v. распределённой по Бернулли с параметром $p=\pr A$}:
\(I_A\sim\Bern(\pr A)\). Обратно, любая случайная величина с
распределением Бернулли является индикатором некоторого события.

\subsubsection{Def. (Схема
Бернулли)}\label{def.-ux441ux445ux435ux43cux430-ux431ux435ux440ux43dux443ux43bux43bux438}

Пусть теперь эксперимент состоит в последовательном проведении \(n\)
экспериментов Бернулли с одинаковой вероятностью успеха \(p\), а
результат записывается в виде строки
\((a_1, a_2, \ldots, a_n), a_i\in \{0,1\}\).

Пространство \(\Omega\) элементарных событий имеет вид
\(\Omega = \{ \omega = (a_1, a_2, \ldots, a_n), a_i\in \{0,1\} \}\).
Каждому исходу \(\omega\) соответствует ``вес''
\({p_\omega = p^k (1-p)^{n-k}}\) \(k=\sum_j a_j\) --- число ``успехов'',
\(n-k\) --- число ``неудач''.

Рассмотрим событие, состоящее в получении \(k\) ``успехов'' и r.v. \(X\)
--- число ``успехов''. Вес каждого такого события ---
\({p_\omega = p^k (1-p)^{n-k}}\). Всего различных таких событий ---
\(n \choose k\). Вероятность этого события ---
\(\pr\{X=k\} = {n \choose k} {p_\omega = p^k (1-p)^{n-k}}\). Говорят,
что рассматриваемая случайная величина \(X\) имеет \emph{биномиальное
распределение} \(\{ \pr A_0, \pr A_2, \ldots, \pr A_n \}\), задаваемое
функций масс \(k \mapsto \pr\{X=k\}\).

\subsubsection{Формализация}\label{ux444ux43eux440ux43cux430ux43bux438ux437ux430ux446ux438ux44f}

\({\Omega = \{ \omega = (a_1, a_2, \ldots, a_n), a_i\in \{0,1\} \}}\).\newline
\({\SigmaField = S(\Omega) = \{ A : A\subset\Omega\}}\).\newline
\({\pr(\{\omega\}) := p_\omega = p^k (1-p)^{n-k}}\). Отсюда в виду
аддитивности \(\pr\): \newline
\({\pr(A) = \sum_{\omega\in A} \pr \{\omega\} = \sum_{\omega\in A} p_{\omega}}\)

Рассмотрим события \(A_k = \{\omega: a_k = 1\}\),
\(\bar A_k = \{\omega: a_k = 0\}\).\newline
Введём \(\sigma\)-алгебры
\(\SigmaField_k = \{ A_k, \bar A_k, \emptyset, \Omega \}\).

Ясно, что \(\pr A_k = p\), \(\pr \bar A_k = 1-p\) и, при
\(i\neq j\),\newline
\(\pr A_i A_j = \pr A_i \pr A_j = p^2\),\newline
\(\pr A_i \bar A_j = \pr A_i \pr\bar A_j = pq\),\newline
\(\pr \bar A_i \bar A_j = \pr\bar A_i \pr\bar A_j = q^2\).

Аналогично, \(A_1, A_2, \ldots, A_n\) --- независимы. Отсюда, алгебры
\(\SigmaField_1, \ldots \SigmaField_n\) --- независимы.

Вероятностное пространство \((\Omega,\SigmaField,\pr)\) имеет
\emph{структуру прямого произведения вероятностных пространств}:

Если даны вероятностные пространства
\((\Omega_1, \SigmaField_1, \pr_1), \ldots (\Omega_n, \SigmaField_n, \pr_n)\).
Их прямое произведение --- вероятностное пространство
\((\Omega, \SigmaField, \pr)\), в котором

\begin{enumerate}
\def\labelenumi{\arabic{enumi}.}

\item
  \(\Omega = \prod_{j=1}^n \Omega_j = \{ \omega=(\omega_1,\ldots,\omega_n): \omega_i\in\Omega_i, i=\overline{1,n} \}\)
\item
  \(\SigmaField = \prod_{j=1}^n \SigmaField_j\) --- минимальная
  \(\sigma\)-алгебра, порождённая цилиндрами
  \(\prod_{j=1}^n A_j, A_j\in\SigmaField_j\).
\item
  \(\pr = \prod_{j=1}^n \pr_j\)
\end{enumerate}

Возвращаясь к нашему частному случаю,

\[\begin{aligned}
  & \Omega_j = \{0,1\} \\
  & \SigmaField_j=\{ \{0\}, \{1\}, \emptyset, \Omega_j \} \\
  & \pr\{1\} = p \\
  & \pr\{0\} = 1-p \\
  & \Omega = \{ \omega=(a_1,\ldots,a_n): a_i\in\Omega_j, j=\overline{1,n} \} \\
  & \SigmaField = \{ A = A_1\times A_2\times \ldots\times A_n : A_j\in\SigmaField_j \} \\
  & p_\omega = p_1(a_1) p_2(a_2) \ldots p_n(a_n) \\
  & \pr A = \sum\limits_{a_j\in\SigmaField_j, j=\overline{1,n}} p_1(a_1)\ldots p_n(a_n) \\
  & \pr\{\omega\} = p^k (1-p)^{n-k}
  \end{aligned}\]

Это и описывает схему Бернулли.
