\subsection{Случайные величины. Дискретные и непрерывные случайные
величины. Смешанные случайные
величины}\label{ux441ux43bux443ux447ux430ux439ux43dux44bux435-ux432ux435ux43bux438ux447ux438ux43dux44b.-ux434ux438ux441ux43aux440ux435ux442ux43dux44bux435-ux438-ux43dux435ux43fux440ux435ux440ux44bux432ux43dux44bux435-ux441ux43bux443ux447ux430ux439ux43dux44bux435-ux432ux435ux43bux438ux447ux438ux43dux44b.-ux441ux43cux435ux448ux430ux43dux43dux44bux435-ux441ux43bux443ux447ux430ux439ux43dux44bux435-ux432ux435ux43bux438ux447ux438ux43dux44b}

Пусть \((\Omega, \SigmaField, \pr)\) --- пространство с вероятностью.

\subsubsection{Def. (Разбиние, индуцированное
функцией)}\label{def.-ux440ux430ux437ux431ux438ux43dux438ux435-ux438ux43dux434ux443ux446ux438ux440ux43eux432ux430ux43dux43dux43eux435-ux444ux443ux43dux43aux446ux438ux435ux439}

Всякая функция \(X : \Omega\mapsto\Omega\prime\) индуцирует разбиение
пространства \(\Omega\), образованное прообразами
\(X^{-1}(\omega\prime)\) точек \(\omega\prime\in\Omega\prime\). Функия
\(X\) называется \emph{постоянной} на множестве
\(X^{-1}(\omega\prime)\). Если индуцированное разбиение конечно или
счётно, то функция \(X\) называется, соответственно, конечнозначной или
счётнозначной.

Всякую счётнозначную функцию \(X\) можно записать в виде
\(X = \sum_{j\in J} I_{A_j} \omega\prime_j\)

Где \(I_A\) --- индикатор. \[
  I_A(\omega) =
  \left\{
  \begin{aligned}
    & 0 & \omega \notin A \\
    & 1 & \omega \in A
  \end{aligned}
  \right.
\]

\subsubsection{Def. (Счётнозначная измеримая
функция)}\label{def.-ux441ux447ux451ux442ux43dux43eux437ux43dux430ux447ux43dux430ux44f-ux438ux437ux43cux435ux440ux438ux43cux430ux44f-ux444ux443ux43dux43aux446ux438ux44f}

Счётнозначной \emph{измеримой} функцией называется функция вида
\hbox{$X = \sum_j I_{A_j}\omega\prime_j$}, где множества \(A_j\)
измеримы (\(A_j \in \SigmaField\)).

Пусть даны два измеримых пространства \((\Omega, \SigmaField)\),
\((\Omega\prime, \SigmaField\prime)\). Счётнозначная функция
\hbox{$X = \sum_j I_{A_j}\omega\prime_j, \quad A_j\in\SigmaField$}
называется \emph{элементарной функцией}, а если функция ещё и
\emph{конечнозначна}, то она называется \emph{простой}.

\subsubsection{Def. (Элементарная случайная
величина)}\label{def.-ux44dux43bux435ux43cux435ux43dux442ux430ux440ux43dux430ux44f-ux441ux43bux443ux447ux430ux439ux43dux430ux44f-ux432ux435ux43bux438ux447ux438ux43dux430}

\emph{Элементарной случайной величиной} называется конечная числовая
\emph{измеримая} счётнозначная функция \(X : \Omega\mapsto\RR\).
Конечнозначная случайная величина называется \emph{простой}.

Такое задание случайной величины даёт представление о её распределении в
\((\Omega, \SigmaField)\). На практике же часто возникает вопрос о
распределении случайной величины на множестве её значений.

Пусть \((\Omega, \SigmaField, \pr)\) --- вероятностное пространство,
\(X : \Omega\mapsto\tilde{\mathcal X}\) --- элементарная случайная
величина, \(\mathcal X = \{ x_1, x_2, \ldots \}\) --- множество её
значений. Рассмотрим вероятность \(\pr_X : S(\mathcal X)\mapsto [0,1]\),
индуцируемую на \(\mathcal X\) случайной величиной \(X\) формулой
\(\pr_X(A) = \pr \{\omega: X(\omega)\in A\}, A\in S(\mathcal X)\)

Значения этих вероятностей определяются вероятнстями
\(\pr_X (x_j) = \pr \{ \omega : X(\omega)=x_j \}, \quad x_j\in \mathcal X\)

\subsubsection{Def. (Распределение элементарной случайной
величины)}\label{def.-ux440ux430ux441ux43fux440ux435ux434ux435ux43bux435ux43dux438ux435-ux44dux43bux435ux43cux435ux43dux442ux430ux440ux43dux43eux439-ux441ux43bux443ux447ux430ux439ux43dux43eux439-ux432ux435ux43bux438ux447ux438ux43dux44b}

Числовая последовательность \(( \pr_X(x_1), \pr_X(x_2), \ldots )\)
называется \emph{распределением} вероятностей случайной величины \(X\).

\subsubsection{Пример}\label{ux43fux440ux438ux43cux435ux440}

Случайная величина \(X : \Omega\mapsto\RR\) принимающая лишь два
значения --- \(0\) или \(1\) --- имеет распределение Бернулли. Для неё
\(\pr_X (x) = p^x (1-p)^{1-x}\).

\subsubsection{Def. (Функция распределения элементарной случайной
величины)}\label{def.-ux444ux443ux43dux43aux446ux438ux44f-ux440ux430ux441ux43fux440ux435ux434ux435ux43bux435ux43dux438ux44f-ux44dux43bux435ux43cux435ux43dux442ux430ux440ux43dux43eux439-ux441ux43bux443ux447ux430ux439ux43dux43eux439-ux432ux435ux43bux438ux447ux438ux43dux44b}

Функцией распределения элементарной случайной величины \(X\) называется
функция \(F_X : \mathcal X\mapsto [0,1]\), определяемая формулой
\[ F_X (x) = \pr \{\omega: X(\omega) \leq x \} = \sum\limits_{x_i \leq x} \pr_X (x_i)\]
\[ \pr_X (x_i) = F_X (x_i) - F_X(x_i - 0)\] Где
\(F_X (x_i - 0) = \lim\limits_{t\uparrow x_i} F_X (t)\)

\begin{center}\rule{0.5\linewidth}{\linethickness}\end{center}

Есть разные способы расширить определение измеримой функции и классы
измеримых по этим определениям функций, вообще говоря, не совпадают.

Заметим, что элементы \(\sigma\)-алгебры, индуцированной отображением
\(X\) измеримы (\(\in\SigmaField\)). Расширяя это свойство:

\subsubsection{Def.1 (Измеримая
функция)}\label{def.1-ux438ux437ux43cux435ux440ux438ux43cux430ux44f-ux444ux443ux43dux43aux446ux438ux44f}

\emph{Измеримой} называется функция \(X : \Omega\mapsto\Omega\prime\),
относительно которой прообразы измеримых множеств измеримы.

\subsubsection{Def.2 (Измеримая функция --- альтернативное
определение)}\label{def.2-ux438ux437ux43cux435ux440ux438ux43cux430ux44f-ux444ux443ux43dux43aux446ux438ux44f-ux430ux43bux44cux442ux435ux440ux43dux430ux442ux438ux432ux43dux43eux435-ux43eux43fux440ux435ux434ux435ux43bux435ux43dux438ux435}

Если в \(\Omega\prime\) введено понятие предела, то \emph{измеримыми
функциями}, в смысле этого предела, называются пределы сходящихся
последовательностей простых функций (или равномерные пределы
элементарных функций).

Пусть дано пространство с вероятностью \((\Omega, \SigmaField, \pr)\)

\subsubsection{TODO}\label{todo}

\begin{itemize}

\item
  \(\RR\)

  \begin{itemize}
  
  \item
    Борелева алгебра
  \item
    Измеримость
  \item
    Эквивалентность двух определений в случае Борелевой алгебры
  \end{itemize}
\end{itemize}

\subsubsection{Def. (Случайная
величина)}\label{def.-ux441ux43bux443ux447ux430ux439ux43dux430ux44f-ux432ux435ux43bux438ux447ux438ux43dux430}

\emph{Случайной величиной} называется конечная \emph{числовая измеримая}
функция \(X : \Omega\mapsto\RR\)

\subsubsection{Def. (Функция
распределения)}\label{def.-ux444ux443ux43dux43aux446ux438ux44f-ux440ux430ux441ux43fux440ux435ux434ux435ux43bux435ux43dux438ux44f}

\begin{center}\rule{0.5\linewidth}{\linethickness}\end{center}

\subsubsection{Дискретные и непрерывные случайные
величины}\label{ux434ux438ux441ux43aux440ux435ux442ux43dux44bux435-ux438-ux43dux435ux43fux440ux435ux440ux44bux432ux43dux44bux435-ux441ux43bux443ux447ux430ux439ux43dux44bux435-ux432ux435ux43bux438ux447ux438ux43dux44b}

\paragraph{Def. (Дискретная случайная
величина)}\label{def.-ux434ux438ux441ux43aux440ux435ux442ux43dux430ux44f-ux441ux43bux443ux447ux430ux439ux43dux430ux44f-ux432ux435ux43bux438ux447ux438ux43dux430}

\emph{Дискретная случайная величина} \(:=\) элементарная случайная
величина. Определяется \emph{функцией распределения масс} (PMF).

\paragraph{Def. (Непрерывная случайная
величина)}\label{def.-ux43dux435ux43fux440ux435ux440ux44bux432ux43dux430ux44f-ux441ux43bux443ux447ux430ux439ux43dux430ux44f-ux432ux435ux43bux438ux447ux438ux43dux430}

\(:=\) случайная величина, определяемая непрерывной функцией плотности
распределения вероятности.

\subsubsection{Способы задания распределения случайной
величины}\label{ux441ux43fux43eux441ux43eux431ux44b-ux437ux430ux434ux430ux43dux438ux44f-ux440ux430ux441ux43fux440ux435ux434ux435ux43bux435ux43dux438ux44f-ux441ux43bux443ux447ux430ux439ux43dux43eux439-ux432ux435ux43bux438ux447ux438ux43dux44b}

\paragraph{Дискретная случайная
величина}\label{ux434ux438ux441ux43aux440ux435ux442ux43dux430ux44f-ux441ux43bux443ux447ux430ux439ux43dux430ux44f-ux432ux435ux43bux438ux447ux438ux43dux430}

\subparagraph{Ряд
распределения}\label{ux440ux44fux434-ux440ux430ux441ux43fux440ux435ux434ux435ux43bux435ux43dux438ux44f}

\subparagraph{Многоугольник
распределения}\label{ux43cux43dux43eux433ux43eux443ux433ux43eux43bux44cux43dux438ux43a-ux440ux430ux441ux43fux440ux435ux434ux435ux43bux435ux43dux438ux44f}

\paragraph{Непрерывная случайная
величина}\label{ux43dux435ux43fux440ux435ux440ux44bux432ux43dux430ux44f-ux441ux43bux443ux447ux430ux439ux43dux430ux44f-ux432ux435ux43bux438ux447ux438ux43dux430}

\subparagraph{Функция
распределения}\label{ux444ux443ux43dux43aux446ux438ux44f-ux440ux430ux441ux43fux440ux435ux434ux435ux43bux435ux43dux438ux44f}

\subparagraph{Плотность
распределения}\label{ux43fux43bux43eux442ux43dux43eux441ux442ux44c-ux440ux430ux441ux43fux440ux435ux434ux435ux43bux435ux43dux438ux44f}
