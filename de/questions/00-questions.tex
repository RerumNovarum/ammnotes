\begin{enumerate}
\def\labelenumi{\arabic{enumi}.}

\item
  Основные понятия:

  \begin{itemize}
  
  \item
    Дифференциальные уравнения
  \item
    Решение дифференциального уравнения
  \item
    Общее решение
  \item
    Общий интеграл
  \item
    Геометрическая интерпретация
  \item
    Задача Коши
  \item
    Изоклины
  \end{itemize}
\item
  Теорема существования и единственности для скалярного уравнения
  (формулировка. Пример неединственности
\item
  Задача о распаде радиоактивного вещества
\item
  Уравнение с разделяющимися переменными. Однородное уравнение
\item
  Линейное дифференциальное уравнение первого порядка
\item
  Уравнение Бернулли
\item
  Уравнение Риккати
\item
  Уравнение в полных дифференциалах
\item
  Необходимый и достаточный признак уравнения в полных дифференциалах
\item
  Интегрирующий множитель
\item
  Система дифференциальных уравнений
\item
  Комплексные решения. Теорема сущестования и единственности для систем
  (формулировка)
\item
  Теорема существования и единственности для уравнения \(n\)-го порядка
  и для линейных систем дифференциальных уравнений
\item
  Функция \(e^x\) и её свойства
\item
  Линейное дифференциальное уравнение \(n\)-го порядка. Свойства
  многочленов от \(p\).
\item
  Общее решение линейного однородного дифференциального уравнения
  \(n\)-го порядка с постоянными коэффициентами (случай простых корней)
\item
  Необходимый и достаточный признак \(k\)-кратного корня многочлена
\item
  Общее решение линейного однородного дифференциального уравнения
  \(n\)-го порядка с постоянными коэффициентами (случай кратных корней)
\item
  Выделение вещественных корней. Математический маятник
\item
  Устойчивые многочлены. Оценка решений уравнений с устойчивым
  характеристическим многочленом
\item
  Устойчивость многочленов \(1\)-го и \(2\)-го порядков. Необходимый и
  достаточный критерий устойчивости вещественного многочлена
\item
  Критерий Рауса-Гурвица. Устойчивость многочлена третьего порядка
\item
  Линейное неоднородное дифференциальное уравнение \(n\)-го порядка.
  Структура общего решения. Квазиполином. Структура общего решения с
  правой частью в виде квазиполинома
\item
  Частные решения уравнения со специальной правой частью
\item
  Метод комплексных амплитуд
\item
  Линейное дифференциальное уравнение \(n\)-го порядка с переменными
  коэффициентами. Линейное однородное уравнение и его свойства. Линейная
  зависимость функций
\item
  Вронскиан и его применение для определения линейной зависимости
  решений линейных дифференциальных уравнений
\item
  Фундаментальная система решений и её свойства
\item
  Восстановление линейного дифференциального уравнения по его
  фундаментальной системе. Формула Остроградского-Лиувилля
\item
  Понижение порядка дифференциального уравнения
\item
  Метод вариации произвольных постоянных
\item
  Двухточечная кравевая задача и её преобразования
\item
  Построение функции Грина и вывод её свойств
\item
  Необходимое и достаточное условие существования функции Грина. Задача
  о собственных значениях краевой задачи
\item
  \(\varepsilon\)-решения. Существование \(\varepsilon\)-решений.
  Ломаные Эйлера
\item
  Теорема Пеано. Теорема единственности решения
\end{enumerate}
