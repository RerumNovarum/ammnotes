\subsection{Формула
Лиувилля}\label{ux444ux43eux440ux43cux443ux43bux430-ux43bux438ux443ux432ux438ux43bux43bux44f}

\subsubsection{Def. (След
матрицы)}\label{def.-ux441ux43bux435ux434-ux43cux430ux442ux440ux438ux446ux44b}

Выражение \(\Tr A = \sum_{j=1}^n a_{j j}\) называется \textbf{следом
матрицы}. (\(\Tr\) for \emph{trace})

Альтернативная нотация: \(\Sp A\) (ger. \emph{Spur}).

\subsubsection{Th. (Формула
Лиувилля)}\label{th.-ux444ux43eux440ux43cux443ux43bux430-ux43bux438ux443ux432ux438ux43bux43bux44f}

Для определителя вронского \(\W\) системы решений ЛОДУ (1) справедлива
\textbf{формула Лиувилля}:
\[ \W = \W(t_0) \exp(\int_{t_0}^t \Tr A(s) \dd s) \]

\begin{proof}
  Найдём $\dot\W$:
  $$ \dot\W =
  \begin{vmatrix}
  & \dot\phi_{1 1} & \cdots & \dot\phi_{1 n} \\
  & \vdots     & \ddots & \vdots     \\
  & \phi_{n 1} & \cdots & \phi_{n n} \\
  \end{vmatrix}
  + \cdots + 
  \begin{vmatrix}
  & \phi_{1 1} & \cdots & \phi_{1 n} \\
  & \vdots     & \ddots & \vdots     \\
  & \dot\phi_{n 1} & \cdots & \dot\phi_{n n} \\
  \end{vmatrix}
  $$

Заметим:
$$\begin{aligned}
\dot x_j = \sum_{j=1}^n a_{i j} x_j, \quad j=\overline{1,n}\\
\cdots\\
\phi_{i k} = \sum_{j=1}^n a_{i j} \phi_{j k}
\end{aligned}$$

Следовательно:
$$ \dot\W = 
\begin{vmatrix}
& \sum_{j=1}^n a_{1 j} \phi_{j 1} & \cdots & \sum_{j=1}^n a_{1 j} \phi_{j n} \\
& \phi_{2 1} & \cdots & \phi_{2 n} \\
& \vdots     & \ddots & \vdots \\
& \phi_{n 1} & \cdots & \phi_{n n}
\end{vmatrix}
 + \cdots + 
\begin{vmatrix}
& \phi_{1 1} & \cdots & \phi_{1 n} \\
& \vdots     & \ddots & \vdots     \\
& \phi_{n-1, 1} & \cdots & \phi_{n-1, n} \\
& \sum_{j=1}^n a_{n j} \phi_{j 1} & \cdots & \sum_{j=1}^n a_{n j} \phi_{j n}
\end{vmatrix}
$$

Теперь в первом определителе

1.  Прибавим к первой строке вторую, домноженную на $-a_{1 2}$ \\
2.   Прибавим к первой строке третью, домноженную на $-a_{1 3}$ \\
3.   $\cdots$ \\
n.   Прибавим к первой строке $n$-й строку, домноженную на $-a_{1 n}$

Аналогично поступим с остальными определителями.
В результате останется:
$$\begin{aligned}
 \dot\W  &=
&& \begin{vmatrix}
 a_{1 1} \phi_{1 1} & \cdots & a_{1 1} \phi_{1 n} \\
     \vdots & \ddots & \vdots \\
     \phi_{n 1} & \cdots & \phi_{n n}
    \end{vmatrix}
&& + \cdots +
&& \begin{vmatrix}
 \phi_{1 1} & \cdots & \phi_{1 n} \\
     \vdots     & \ddots & \vdots \\
     a_{n n} \phi_{n 1} & \cdots & a_{n n} \phi_{n n} 
    \end{vmatrix}\\
& =
a_{1 1} 
&& \begin{vmatrix}
 \phi_{1 1} & \cdots & \phi_{1 n} \\
     \vdots & \ddots & \vdots \\
     \phi_{n 1} & \cdots & \phi_{n n}
    \end{vmatrix}
&& + \cdots +
a_{n n} 
&& \begin{vmatrix}
 \phi_{1 1} & \cdots & \phi_{1 n} \\
       \vdots     & \ddots & \vdots \\
       \phi_{n 1} & \cdots & \phi_{n n}
      \end{vmatrix}
= \sum_{j=1}^n a_{j j} \W
\end{aligned}$$
Это скалярное линейное Д.У. первого порядка, его решение имеет вид:

$$\W(t) = \W(t_0) \exp(\int_{t_0}^t \Tr A(s) \dd s)$$
\end{proof}

\subsection{Матричное дифференциальное
уравнение}\label{ux43cux430ux442ux440ux438ux447ux43dux43eux435-ux434ux438ux444ux444ux435ux440ux435ux43dux446ux438ux430ux43bux44cux43dux43eux435-ux443ux440ux430ux432ux43dux435ux43dux438ux435}

Наряду с векторными уравнениями \(\dot x = A x\) рассмотрим матричные
уравнения: \[\begin{aligned}
& \dot X = A X \\
& X : (q_1, q_2) \mapsto \RR^{n\times n} \\
& A : (q_1, q_2) \mapsto \RR^{n\times n}
\end{aligned}\]

\subsection{Связь между векторными и матричными
Д.У.}\label{ux441ux432ux44fux437ux44c-ux43cux435ux436ux434ux443-ux432ux435ux43aux442ux43eux440ux43dux44bux43cux438-ux438-ux43cux430ux442ux440ux438ux447ux43dux44bux43cux438-ux434.ux443.}

Пусть дано матричное уравнение \[ \dot X = A X \] Матрицу \(X\) запишем
в виде \(X = (X_1, \ldots, X_n)\), где \(X_j\) --- \(j\)-й столбец
матрицы \(X\). Матричное уравнение теперь можно записать в виде: \[
(\dot X_1, \ldots, \dot X_n) = (A X_1, \ldots, A X_n)
\] Тогда матричная функция будет являться решением данного матричного
уравнения тогда и только тогда, когда её векторы-столбцы будут являться
решениями векторного Д.У.

\subsection{Фундаментальная
матрица}\label{ux444ux443ux43dux434ux430ux43cux435ux43dux442ux430ux43bux44cux43dux430ux44f-ux43cux430ux442ux440ux438ux446ux430}

\subsubsection{Def. (Фундаментальная
матрица)}\label{def.-ux444ux443ux43dux434ux430ux43cux435ux43dux442ux430ux43bux44cux43dux430ux44f-ux43cux430ux442ux440ux438ux446ux430}

Матричная функция \(\Phi\) называется фундаментальной матрицей ЛОДУ
\(\dot x = A x\), если её векторы-столбцы образуют ФСР этого уравнения

Очевидно, что для того чтобы \(\Phi\) являлась фундаментальной матрицей,
чтобы она была решением соответствующего матричного уравнения и была
невырождена \(\forall t\in (q_1, q_2)\)

\begin{proof}
Необходимость:

Пусть $\Phi$ фундаментальная, тогда её векторы столбцы образуют ФСР,
  а значит они являются решениями и составленный из них определитель
  есть определитель Вронского, который не обращается в нуль,
  т.е. матрица невырожденна.

Достаточность:

Пусть $\Phi$ невырождена, а её столбцы являются решениями.
Тогда её столбцы составляют систему из $n$ линейно-независимых решений и образуют ФСР.
\end{proof}

\subsubsection{Свойства фундаментальной
матрицы}\label{ux441ux432ux43eux439ux441ux442ux432ux430-ux444ux443ux43dux434ux430ux43cux435ux43dux442ux430ux43bux44cux43dux43eux439-ux43cux430ux442ux440ux438ux446ux44b}

Если \(\Phi\) --- фундаментальная, а \(C\in\RR^{n\times n}\)
невырожденная числовая матрица, То: \[\Psi = \Phi C\] также является
фундаментальной матрицей этого уравнения

\begin{proof}
  Требуется показать, что $\Psi$ является решением и невырождена:
  $$\begin{aligned}
  \det \Phi C = \det \Phi \det C \neq 0 \\
  \dot \Psi = \dot \Phi C = A \Phi C = A \Psi
  \end{aligned}$$
  \end{proof}

\begin{center}\rule{0.5\linewidth}{\linethickness}\end{center}
