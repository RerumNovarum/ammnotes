\section{Системы линейных дифференциальных
уравнений}\label{ux441ux438ux441ux442ux435ux43cux44b-ux43bux438ux43dux435ux439ux43dux44bux445-ux434ux438ux444ux444ux435ux440ux435ux43dux446ux438ux430ux43bux44cux43dux44bux445-ux443ux440ux430ux432ux43dux435ux43dux438ux439}

\subsection{Def. (Линейная система дифференциальных
уравнений)}\label{def.-ux43bux438ux43dux435ux439ux43dux430ux44f-ux441ux438ux441ux442ux435ux43cux430-ux434ux438ux444ux444ux435ux440ux435ux43dux446ux438ux430ux43bux44cux43dux44bux445-ux443ux440ux430ux432ux43dux435ux43dux438ux439}

Система \[
\left\{ \begin{aligned}
  & \dot x_1(t) &&= a_{1 1}(t) x_1(t) + a_{1 2}(t) x_2(t) + \ldots + a_{1 n}(t) x_n(t) + b_1(t) \\
  & \dot x_2(t) &&= a_{2 1}(t) x_1(t) + a_{2 2}(t) x_2(t) + \ldots + a_{2 n}(t) x_n(t) + b_2(t) \\
  & \vdots && \\
  & \dot x_n(t) &&= a_{n 1}(t) x_1(t) + a_{n 2}(t) x_2(t) + \ldots + a_{n n}(t) x_n(t) + b_n(t)
\end{aligned} \right.
\]

Здесь:

\(a_{i j} : (q_1, q_2) \mapsto \RR, \quad i,j = \overline{1,n}\),

\(b_i \in \RR \quad i=\overline{1,n}\) --- заданные коэффициенты, а

\(x_i : (q_1, q_2) \mapsto \RR, \quad i=\overline{1,n}\) --- искомые
функции,

называется \textbf{линейной системой дифференциальных уравнений с
переменными коэффициентами}.

\subsubsection{В векторном
виде:}\label{ux432-ux432ux435ux43aux442ux43eux440ux43dux43eux43c-ux432ux438ux434ux435}

\[\begin{aligned}
& \dot{\mathbf x}(t) = A(t) \mathbf x(t) + \mathbf{b}(t) \\
& \mathbf b =
  \begin{pmatrix}
    b_1 \\
    b_2 \\
    \vdots \\
    b_n
    \end{pmatrix},\quad 
  \mathbf x =
  \begin{pmatrix}
    x_1 \\
    x_2 \\
    \vdots \\
    x_n
    \end{pmatrix} \\
& A = 
  \begin{pmatrix}
      & a_{1 1} & a_{1 2} & \ldots & a_{1 n} & \\
      & a_{2 1} & a_{2 2} & \ldots & a_{2 n}   \\
      & \vdots  & \vdots  & \ddots & \vdots    \\
      & a_{n 1} & a_{n 2} & \cdots & a_{n n}
      \end{pmatrix} \\
& \begin{aligned}
& \mathbf x &&: (q_1, q_2) && \mapsto \RR^n \\
& \mathbf b &&: (q_1, q_2) && \mapsto \RR^n \\
& A &&: (q_1, q_2) && \mapsto \RR^{n\times n}
\end{aligned}
\end{aligned}\]

\subsubsection{Nota bene:}\label{nota-bene}

Далее это уравнение будем записывать в виде \[\dot x = A x + b\],
подразумевая тождественное равенство функций на некотором промежутке

\subsection{Def. (ЛНСДУ)}\label{def.-ux43bux43dux441ux434ux443}

Линейная система называется неоднородной, если \(\sum_j |b_j| \neq 0\)
(где \(0\) --- функция, тождественно равная нулю)

\subsection{Def. (ЛОСДУ)}\label{def.-ux43bux43eux441ux434ux443}

Линейная система называется однородной, если \(\sum_j |b_j| \equiv 0\)

\subsection{Def. (Решение
ЛСДУ)}\label{def.-ux440ux435ux448ux435ux43dux438ux435-ux43bux441ux434ux443}

Решение линейной системы дифференциальных уравнений \(\dot x = A x + b\)
называется векторная функция \(\phi : (q_1, q_1) \mapsto \RR^n\),
определённая на некотором интервале \((q_1, q_2)\), обращающая уравнение
в тождество на этом интервале.

\subsection{Def. (Общее
решение)}\label{def.-ux43eux431ux449ux435ux435-ux440ux435ux448ux435ux43dux438ux435}

Общим решением системы называется семейство функций
\(\phi_{c_1, c_2, \ldots, c_n}\), зависящих от параметров
\(c_1, c_2, \ldots, c_n\), такое что
\(\forall c_1, \ldots, c_n \quad \phi_{c_1 \ldots c_n}\) является
решением, и \(\forall\) решения \(\phi\)
\(\exists c_1, \ldots, c_n: \quad \phi = \phi_{c_1 \ldots c_n}\).

\subsection{Def. (Задача Коши для
ЛСДУ)}\label{def.-ux437ux430ux434ux430ux447ux430-ux43aux43eux448ux438-ux434ux43bux44f-ux43bux441ux434ux443}

Найти решение системы \(\dot x = A x + b\), удовлетворяющее
\emph{начальным условиям} \[
\mathbf x(t_0) = \mathbf x_0 =
{\begin{pmatrix}
x_0^0 & x_1^0 & \cdots & x_n^0
\end{pmatrix}}^{\mathtt T}\]

\subsection{Th. (Формулировка теоремы существования и
единственности)}\label{th.-ux444ux43eux440ux43cux443ux43bux438ux440ux43eux432ux43aux430-ux442ux435ux43eux440ux435ux43cux44b-ux441ux443ux449ux435ux441ux442ux432ux43eux432ux430ux43dux438ux44f-ux438-ux435ux434ux438ux43dux441ux442ux432ux435ux43dux43dux43eux441ux442ux438}

Если \(A\) и \(b\) непрерывны на \((q_1, q_2)\),

то \(\forall t_0 \in (q_1, q_2) \quad\forall x_0 \in \RR^n\)

задача Коши \[\left\{\begin{aligned}
& \dot x &&= A x + b \\
& x(t_0) &&= x_0
\end{aligned}\right.\]

имеет единственное решение, определённое на \((q_1, q_2)\).
