\documentclass{article}

\usepackage{hyperref,amsthm,amsmath,amssymb,fontspec}
\setmainfont{CMU Serif}

\title{Конспекты к экзамену по математической статистике}

\providecommand{\SigmaField}{\mathcal F}
\providecommand{\FF}{\SigmaField}
\providecommand{\BorelField}{\mathcal B}
\providecommand{\BB}{\BorelField}
\providecommand{\RR}{\mathbb R}
\providecommand{\CC}{\mathbb C}
\providecommand{\KK}{\mathbb K}
\providecommand{\Expectation}{\mathbb E}
\providecommand{\E}{\Expectation}
\providecommand{\variance}{\mathbb D}
\providecommand{\D}{\variance}
\providecommand{\Probability}{\mathbb P}
\providecommand{\pr}{\Probability}
\providecommand{\PrSpace}{(\Omega, \SigmaField, \pr)}
\providecommand{\Bern}{\mathrm{Bern}}
\providecommand{\Gaussian}{\mathcal N}
\providecommand{\dd}{\mathrm d}
\providecommand{\dx}{\dd x}
\providecommand{\dmu}{\dd\mu}
\providecommand{\dF}{\dd F}
\providecommand{\const}{\mathtt{const}}
\providecommand{\diam}{\mathrm{diam}}
\providecommand{\cov}{\mathrm{cov}}
\providecommand{\pop}{\mathcal L}

\theoremstyle{definition}
\newtheorem{dfn}{Def.}
\theoremstyle{plain}
\newtheorem{thm}{Thm.}
\theoremstyle{remark}
\newtheorem{nb}{NB:}

\setcounter{subsection}{-1}

\begin{document}
\maketitle
\vfill
\tableofcontents
\newpage

\begin{enumerate}
\def\labelenumi{\arabic{enumi}.}

\item
  Основные понятия:

  \begin{itemize}
  
  \item
    Дифференциальные уравнения
  \item
    Решение дифференциального уравнения
  \item
    Общее решение
  \item
    Общий интеграл
  \item
    Геометрическая интерпретация
  \item
    Задача Коши
  \item
    Изоклины
  \end{itemize}
\item
  Теорема существования и единственности для скалярного уравнения
  (формулировка. Пример неединственности
\item
  Задача о распаде радиоактивного вещества
\item
  Уравнение с разделяющимися переменными. Однородное уравнение
\item
  Линейное дифференциальное уравнение первого порядка
\item
  Уравнение Бернулли
\item
  Уравнение Риккати
\item
  Уравнение в полных дифференциалах
\item
  Необходимый и достаточный признак уравнения в полных дифференциалах
\item
  Интегрирующий множитель
\item
  Система дифференциальных уравнений
\item
  Комплексные решения. Теорема сущестования и единственности для систем
  (формулировка)
\item
  Теорема существования и единственности для уравнения \(n\)-го порядка
  и для линейных систем дифференциальных уравнений
\item
  Функция \(e^x\) и её свойства
\item
  Линейное дифференциальное уравнение \(n\)-го порядка. Свойства
  многочленов от \(p\).
\item
  Общее решение линейного однородного дифференциального уравнения
  \(n\)-го порядка с постоянными коэффициентами (случай простых корней)
\item
  Необходимый и достаточный признак \(k\)-кратного корня многочлена
\item
  Общее решение линейного однородного дифференциального уравнения
  \(n\)-го порядка с постоянными коэффициентами (случай кратных корней)
\item
  Выделение вещественных корней. Математический маятник
\item
  Устойчивые многочлены. Оценка решений уравнений с устойчивым
  характеристическим многочленом
\item
  Устойчивость многочленов \(1\)-го и \(2\)-го порядков. Необходимый и
  достаточный критерий устойчивости вещественного многочлена
\item
  Критерий Рауса-Гурвица. Устойчивость многочлена третьего порядка
\item
  Линейное неоднородное дифференциальное уравнение \(n\)-го порядка.
  Структура общего решения. Квазиполином. Структура общего решения с
  правой частью в виде квазиполинома
\item
  Частные решения уравнения со специальной правой частью
\item
  Метод комплексных амплитуд
\item
  Линейное дифференциальное уравнение \(n\)-го порядка с переменными
  коэффициентами. Линейное однородное уравнение и его свойства. Линейная
  зависимость функций
\item
  Вронскиан и его применение для определения линейной зависимости
  решений линейных дифференциальных уравнений
\item
  Фундаментальная система решений и её свойства
\item
  Восстановление линейного дифференциального уравнения по его
  фундаментальной системе. Формула Остроградского-Лиувилля
\item
  Понижение порядка дифференциального уравнения
\item
  Метод вариации произвольных постоянных
\item
  Двухточечная кравевая задача и её преобразования
\item
  Построение функции Грина и вывод её свойств
\item
  Необходимое и достаточное условие существования функции Грина. Задача
  о собственных значениях краевой задачи
\item
  \(\varepsilon\)-решения. Существование \(\varepsilon\)-решений.
  Ломаные Эйлера
\item
  Теорема Пеано. Теорема единственности решения
\end{enumerate}

\subsection{Случайная выборка, генеральная совокупность, функция распределения выборки}

\begin{dfn}{Выборка (sample)}
Пусть эксперемент состоит в проведении $n$ испытаний,
результат $j$-го из которых
является случайной величиной $X_j: \Omega_j\to\samplespace_j$.

Кортёж из этих случайных величин (случайный вектор)
$X = (X_1, \dotsc, X_n)$ называется (случайной) выборкой,
а r.v. $X_j$ называются элементами выборки

А значение $x = (x_1, \dotsc, x_n) = X(\omega)$ называется реализацией выборки

Далее всегда, если не указано иное,
случайные величины будут обозначаться заглавными буквами,
а их реализации соответствующими строчными

Далее $X_j$ полагаются независимыми
\end{dfn}

\begin{dfn}{Выборочное пространство (sample space)}
Выборочным пространством
называется измеримое пространство
$(\samplespace, \mathcal A)$,
где $\samplespace = \{X(\omega) ; \omega\in\Omega\}$
есть множество возможных значений выборки,
а $\mathcal A$ --- $\sigma$-алгебра в $\samplespace$
\end{dfn}

Особенно важен случай, когда случайные величины $X_j$
являются независимыми и
имеют распределение одной случайной величины $\xi$.
Этот случай соответствует повторению $n$ раз одного эксперемента,
описываемого случайной величиной $\xi$

\begin{dfn}{Генеральная совокупность (population)}
Генеральной совокупностью
называют распределение $\pop(\xi)$ случайной величины $\xi$

Оно может быть задано, например, множеством возможных значений r.v. $\xi$
и её функцией распределения

При этом $X$ называют выборкой из (генеральной совокупности) $\pop(\xi)$
\end{dfn}

\begin{dfn}{Функция распределения выборки}
 $X\in\pop(\xi)$
$$F_X(x) = \pr\{X\leq x\} = \prod\pr\{X_j\leq x_j\} = \prod F_{X_j}(x_j)$$
\end{dfn}

\subsection{Эмпирическая функция распределения, гистограмма}

Пусть $A\subset\Omega_0$ событие,
происходящее в ходе испытания с вероятностью $\pr A = p$,
и пусть эксперимент состоит в проведении $n$ таких независимых испытаний

Тогда
$$\Omega = \prod_{j=1}^n \Omega_0$$
А случайная величина
$$X_j =
I_\{\omega ; \omega_j \in A\} =
\left\{
\begin{aligned}
1; && \omega_j\in A \\
0; && \omega_j\notin A
\end{aligned}\right.$$
является индикатором того, что в ходе $j$-го испытания случилось событие~$A$

Пусть r.v. $k = \sum_{j=1}^n X_j$ --- число проявлений $A$ в ходе эксперимента. \\
Введём r.v. $p_n^* = \frac{1}{n} \sum_{j=1}^n X_j$. \\
Очевидно $\E p_n^* = p$.\\
Кроме того, из ЗБЧ в форме Бернулли следует
$$\lim_{n\to\infty} \pr\{ |p_n^* - p| < \varepsilon \} = 1 \quad \forall\varepsilon>0$$

Таким образом, значение случайной величины $p_n^*$
можно считать приближённой оценкой величины $p$


Пусть теперь
$X = (X_1, \dotsc, X_n)$ --- выборка объёма $n$ из генеральной совокупности $\pop(\xi)$,
$x = (x_1, \dotsc, x_n)$ --- реализация.

\begin{dfn}{Порядковые статистики}
Каждой реализации $x$ можно сопоставить в соответствие его перестановку
$x_{(1)} \leq \dotsb \leq x_{(n)}$,

$j$-й порядковой статистикой назвается
случайная величина $X_{(j)}$,
при каждой реализации $X(\omega)=x$,
принимает значение $X_{(j)}(\omega) = x_{(j)}$
\end{dfn}

\begin{dfn}{Вариационный ряд}
Случайный вектор $(X_{(1)}, \dotsc, X_{(n)})$ называется вариационным рядом
\end{dfn}

\begin{dfn}{Эмпирическая функция распределения}
Для каждого $t\in\RR$ зададим случайную величину $\mu_n(x)$,
равную количеству элементов выборки $X$,
значения которых не превосходят $t$:
$$\mu_n(x) = \sum I_{ \{ X_j \leq t \} }$$
Эмпирической функцией распределения, построенной по выборке $X$,
называют случайную функцию $F_n: t\mapsto \rvspace(\Omega)$
$$F_n(x) = \frac{1}{n} \mu_n(t)$$
Её значение в точке $t$ является случайной величиной,
сходящейся по вероятности к значению $F(t)$
теоретической функции распределения

\begin{mbox}
EDF можно перезаписать с помощью функции Хевисайда (Heaviside):
\nopagebreak
$$H(t) =
\left\{\begin{aligned}
0; && t < 0 \\
1; && t \geq 0
\end{aligned}\right.$$
$$F_n(t) = \frac{1}{n} \sum_{j=1}^n H(t - X_{(k)})$$
\end{mbox}
\end{dfn}

\begin{dfn}{Гистограмма}
Разобьём область значений r.v. $\xi$ на равные интервалы $\Delta_i$,
и для каждого $\Delta_i$ подсчитаем число $n_i$ элементов $x_j$ вектора $x$,
попавших в $\Delta_i$, $n = \sum n_i$.

Построим график ступенчатой функции $$t\mapsto \frac{n_i}{n h_i},\quad t\in\Delta_i, h_i=|\Delta_i|$$
Полученный график (при желании, само отображение) называется Гистограммой,
построенной по данной реализации выборки

Соединим середины смежных отрезков этого графика.
Полученная ломанная называется полигоном частот

С уменьшением $\max\{h_i\}$,
гистограмма и полигон частот всё более точно приближают
вероятности попадания в каждый из интервалов разбиения
\end{dfn}

\end{document}
