\subsection{Оценки максимального правдоподобия}

Пусть $X=(X_1,\dotsc,X_n)$ --- выборка из $\pop(\xi) = \{ F_\theta; \theta\in\Theta \}$.
При каждом $x\in\samplespace$, $L:\theta\to\ab01$ --- функция правдоподобия для реализации $x$.

\begin{dfn}{Оценка максимального правдоподобия}
Оценкой $\hat\theta$ максимального правдоподобия
называется статистика, каждая реализция $\theta=\hat\theta(\omega)$ которой
является точкой максимума функции $L$ правдоподобия при данной реализации $x=X(\omega)$

$$\hat\theta:\omega\mapsto\sup_{\theta\in\Theta} f(X(\omega);\theta)$$
\end{dfn}

\begin{thm}{Уравнения правдоподобия}
Если $L:\Theta\to\ab01$ дифференцируема
и при каждой реализации $x\in\samplespace$
супремум достигается в внутренней точке множества $\Theta$,
то $$\frac{\partial f}{\partial\theta}(x;\theta) = 0$$
или, то же самое: $$\frac{\partial\ln f}{\partial\theta}(x;\theta) = 0$$
$$\frac{\partial\ln f}{\partial\theta_i}(x;\theta) = 0
 \quad \forall x\in\samplespace\quad i=\overline{1,m}$$

Эти уравнения называются уравнениями максимального правдоподобия,
а значение оценки максимального правдоподобия (о.м.п.) ищется как её решение
при заданной реализации.
\end{thm}
