\subsection{Выборочные характеристики. Выборочные моменты}

Пусть $X = (X_1, \dotsc, X_n)$ --- выборка из $\pop(\xi)$,
$F$ и $F_n$ --- соответственно теоритическая и эмпирическая функции распределения.

Всякой характиристике $\tilde g$ случайной величины $\xi$
$$\tilde g = \int_\RR g(t) \dF(t)$$
можно поставить в соответствие статистический аналог
--- случайную величину $G$:
$$G = \int_\samplespace g(x) \dF_n(x) = \frac{1}{n} \sum_{j=1}^n g\circ X_j$$
$$G(\omega)
= \int_\samplespace g(x) \dd\left((F_n(x))(\omega)\right)
= \frac{1}{n} \sum_{j=1}^n g(x_j)$$

Выборочным моментом $k$-го порядка называется
статистический аналог характеристики $\alpha_k = \E \xi^k = \int_\RR t^k \dF(t)$:

$$A_k = \frac{1}{n} \sum_{j=1}^n X_j^k$$

$\bar X = A_1$ называют выброчным средним.

Выборочным центральным моментом $k$-го порядка
называют случайную величину $M_k$
--- статистический аналог характеристики $\mu_k = \E (\xi - \E \xi)^k = \int_\RR (t - \alpha_1)^k \dF(t)$

$$M_k = \frac{1}{n} \sum_{j=1}^n \left( X_j - \bar X \right)^k$$

$M_2$ называют выборочной дисперсией

\begin{nb}{Выборочное среднее является несмещённой оценкой математического ожидания}
$$\E \bar X = \frac{1}{n}\sum_{j=1}^n \E X_j = \frac{n \alpha_1}{n} = \alpha_1$$
\end{nb}
