\subsection{Функция правдоподобия. Неравенство {Крамера-Рао}}

Зададим класс допустимых распределений r.v. $\xi:\Omega_0\to\samplespace_0\subset\RR$
функцией двух переменных $$f: \RR\times\Theta\to\ab01$$
при каждом фиксированном $\theta$ являющейся
плотностью распределения вероятностей $f_\theta: \RR\to\ab01$
соответствующего значению $\theta$ параметра

Пусть $X = (X_1, \dotsc, X_n)$ --- выборка из $\pop(\xi)$,
$x = X(\omega)$ --- реализация выборки.
Функцией $f_s:\samplespace\times\Theta\to\ab01$
будем обозначать плотность распределения вероятностей выборки
$f_s(x; \theta) = \prod_{j=1}^n f(x_j; \theta)$

\begin{dfn}{Функция правдоподобия (likelihood fuction)}
При фиксированном $x\in\samplespace$
функция $L: \theta\mapsto f_s(x;\theta)$ называется функцией правдоподобия.
При данной реализации $x$, значение $L(\theta)$ характеризует
вероятность получения такой реализации при таком значении параметра,
i.e. ``правдоподобие'' этого значения параметра.

Далее будем считать,
что при любом $x$
отображение $f_s$ дифференцируемо по $\theta$ ($\iff f$ дифференцируемо по $\theta$)
\end{dfn}

\begin{dfn}{Вклад (score) выборки}
Пусть $$u = \frac{\partial\ln f_s}{\partial\theta}:\samplespace\times\Theta\to\RR$$
--- частная производная функции логарифма правдоподобия.
Вкладом выборки $X$
называется случайная функция%
\footnote{Далее случайные функции (аналогично последовательности, векторы)
могут рассматриваться
и как $\RR\to\rvspace$ отображения в пространство случайных величин,
и как $\Omega\to\rvspace(\RR,\RR)$
отображение из пространства элементарных событий в пространство измеримых функций,
и как $\Omega\times\RR\to\RR$
в зависимости от контекста и потребностей}
$U:\Theta\to\rvspace$
$$U(\theta)(\omega) = u(X(\omega);\theta) = u(x;\theta)$$

\begin{itemize}
\item Вклад характеризует чувствительность плотности распределения выборки
к изменению значения параметра $\theta$
\item Кроме того, реализация вклада (выборки в данное значение параметра) характеризует
``степень несоответствия'' данного значения параметра полученной реализации выборки.
\end{itemize}
\end{dfn}

\begin{dfn}{Регулярная статистическая модель}
Статистическая модель,
позволяющая
дифференцировать (всякие $\int L$ и вообще всё что вздумается) по $\theta$,
переставлять операторы интегрирования и дифференцирования,
и разрешающая прочий матан называется регулярной

Далее рассматриваются регулярные модели
\end{dfn}

\begin{thm}{Свойства функции правдоподобия и вклада}
$$\E_\theta L(\theta) = \int_\samplespace f_s(x;\theta) \dx = 1 \quad\forall\theta\in\Theta$$
$$\E_\theta U(\theta) = 0 \quad\forall\theta\in\Theta$$
То есть ожидаемый вклад выборки в истинное значение параметра --- 0.
Это соответствует ожиданию достижения максимума функции (логарифма) правдоподобия в точке $\theta$,
и минимальному "несоответствию" между предполагаемым значением параметра и полученной реализацией выборки.
\end{thm}
\begin{proof}
Первое равенство естественно.

$$\begin{aligned}
\E_\theta U(\theta)
&= \int_\samplespace \frac{\partial\ln f_s}{\partial\theta}(x;\theta) f_s(x;\theta)\dx
 = \int_\samplespace \frac{\partial f_s}{\partial\theta}(x;\theta) \frac1{f_s(x;\theta)} f_s(x;\theta)\dx=\\
&= \int_\samplespace \frac{\partial f_s}{\partial\theta}(x;\theta)\dx
 = \frac{\partial}{\partial\theta} \int_\samplespace f_s(x;\theta)\dx
 = \frac{\partial}{\partial\theta} 1
 = 0
\end{aligned}$$
\end{proof}

\begin{dfn}{Информация Фишера}
\emph{Информацией Фишера}
о параметре $\theta$
содержащейся в выборке $X$
называется функция $\fisherinfo:\Theta\to\RR_+$ (или её значение):
$$\fisherinfo(\theta) = \D_\theta U(\theta) = \E_\theta U^2(\theta) = \int_\samplespace u(x;\theta) f(x;\theta) \dx$$

А функции $\fisherinfo_j$
$$\fisherinfo_j(\theta)
= \D_\theta \frac{\partial\ln f(X_j;\theta)}{\partial\theta}
= \int_\RR \left(\frac{\partial\ln f(\tau;\theta)}{\partial\theta}\right)^2 f(\tau;\theta) \dd \tau $$
количеством информации о параметре $\theta$,
содержащейся в $j$-м наблюдении

$$\fisherinfo(\theta) = \sum_{j=1}^n \fisherinfo_j(\theta) = n\fisherinfo_1(\theta)$$

Информация Фишера
представляет собой ожидание (в предположении о истинности значения $\theta$)
некоего абсолютного значения несоответствия значения $\theta$ параметра
полученной реализации выборки
\end{dfn}

\begin{thm}{Неравенство Крамера-Рао}
Для любой несмещённой оценки $T=t\circ X$ параметра $\theta$ справедливо
$$\D_\theta T \geq \frac{1}{\fisherinfo(\theta)}$$
\end{thm}
\begin{proof}
$T=t\circ X$ --- несмещённая оценка. Значит
$$\E_\theta T = \int_\samplespace t(x)f_s(x;\theta) \dx = \theta
\qquad \left|\frac\partial{\partial\theta}\right.$$
$$\begin{aligned}
1
&=\frac\partial{\partial\theta}\int_\samplespace t(x)f_s(x;\theta)\dx
 = \int_\samplespace t(x) \frac\partial{\partial\theta} f_s(x;\theta)\dx
 = \int_\samplespace t(x) \frac{\partial\ln f_s}{\partial\theta}(x;\theta)f_s(x;\theta)\dx \\
&= \E_\theta(T U(\theta)) = \E_\theta((T - \theta)(U(\theta)-0)) + \underbrace{\theta\E_\theta U(\theta)}_{=0}
 = \cov(T,U(\theta))
 \leq \sqrt{\D_\theta T \D_\theta U(\theta)}
\end{aligned}$$
$$\D_\theta T \geq \frac{1}{\D_\theta U(\theta)} = \frac{1}{\fisherinfo(\theta)}$$
\end{proof}

Если существует оценка, для которой достигается эта нижняя граница,
то, очевидно, эта оценка является оптимальной (и единственной)

\begin{thm}
Если нижняя граница Крамера-Рао достигается,
то ${1 = \cov(T,U(\theta)) = \D_\theta T \D_\theta U(\theta)}$,
i.e. $$\corr(T,U(\theta)) = \frac{\cov(T,U(\theta))}{\D_\theta T \D_\theta U(\theta)} = 1$$
Что возможно тогда и только тогда, когда $T$ и $U(\theta)$ линейно-зависимы,
i.e. $\forall\theta\in\Theta$ существуют константы $a(\theta), b(\theta)$, такие что
$$\pr\{T \neq a(\theta) U(\theta) + b(\theta)\} = 0$$

Причём $\forall\theta\in\Theta$
$$\D_\theta T = \D_\theta(a(\theta)U(\theta) + b(\theta))
= \D_\theta(a(\theta) U(\theta)) = a^2(\theta) \D_\theta U(\theta) 
= \frac{1}{\fisherinfo(\theta)}$$
$$a(\theta) = \pm\frac{1}{\D_\theta U(\theta)}$$
$$\theta=\E_\theta T = b(\theta)$$

$$T = \frac{1}{\D_\theta U(\theta)}U(\theta) + \theta$$
\end{thm}
