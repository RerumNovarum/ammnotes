\subsection{Функция правдоподобия. Неравенство Крамера-Рао}

Зададим класс допустимых распределений r.v. $\xi:\Omega_0\to\samplespace_0\subset\RR$
функцией $$f: \RR\times\Theta\to\ab01$$
при каждом фиксированном $\theta$ являющейся
плотностью распределения вероятностей $f_\theta: \RR\to\ab01$
соответствующего значению $\theta$ параметра

Пусть $X = (X_1, \dotsc, X_n)$ --- выборка из $\pop(\xi)$,
$x = X(\omega)$ --- реализация выборки.
Функцией $f:\samplespace\times\Theta\to\ab01$
будем обозначать плотность распределения вероятностей выборки
$f(x; \theta) = \prod_{j=1}^n f(x_j; \theta)$

\begin{dfn}{Функция правдоподобия (likelihood fuction)}
При фиксированном $x\in\samplespace$
функция $L: \theta\mapsto f(x;\theta)$ называется функцией правдоподобия.

Далее будем считать,
что при любом $x$
отображение $f$ дифференцируемо по $\theta$
\end{dfn}

\begin{dfn}{Вклад (score) выборки}
Пусть $$u = \frac{\partial\ln f}{\partial\theta}:\samplespace\times\Theta\to\RR$$
--- частная производная функции логарифма правдоподобия.
Вкладом выборки $X$
называется случайная функция%
\footnote{Далее случайные функции (аналогично последовательности, векторы)
могут рассматриваться
и как $\RR\to\rvspace$ отображения в пространство случайных величин,
и как $\Omega\to\rvspace(\RR,\RR)$
отображение из пространства элементарных событий в пространство измеримых функций,
и как $\Omega\times\RR\to\RR$
в зависимости от контекста и потребностей}
$U:\Theta\to\rvspace$
$$U(\theta)(\omega) = u(X(\omega);\theta) = u(x;\theta)$$

Вклад характеризует чувствительность плотности распределения выборки
к изменению значения параметра $\theta$
\end{dfn}

\begin{dfn}{Регулярная статистическая модель}
Статистическая модель,
позволяющая
дифференцировать (всякие $\int L$ и вообще всё что вздумается) по $\theta$,
переставлять операторы интегрирования и дифференцирования,
и разрешающая прочий матан называется регулярной

Далее рассматриваются регулярные модели
\end{dfn}

\begin{thm}{Свойства функции правдоподобия и вклада}
$$\E_\theta L(\theta) = \int_\samplespace f(x;\theta) \dx = 1 \quad\forall\theta\in\Theta$$
$$\E_\theta U(\theta) = 0 \quad\forall\theta\in\Theta$$
\end{thm}
\begin{proof}
Первое равенство естественно.

$$\begin{aligned}
\E_\theta U(\theta)
&= \int_\samplespace \frac{\partial\ln f}{\partial\theta}(x;\theta) f(x;\theta)\dx
 = \int_\samplespace \frac{\partial f}{\partial\theta}(x;\theta) \frac1{f(x;\theta)} f(x;\theta)\dx=\\
&= \int_\samplespace \frac{\partial f}{\partial\theta}(x;\theta)\dx
 = \frac{\partial}{\partial\theta} \int_\samplespace f(x;\theta)\dx
 = \frac{\partial}{\partial\theta} 1
 = 0
\end{aligned}$$
\end{proof}

\begin{dfn}{Информация Фишера}
\emph{Информацией Фишера}
о параметре $\theta$
содержащейся в выборке $X$
называется функция $\fisherinfo:\Theta\to\RR_+$ (или её значение):
$$\fisherinfo(\theta) = \D U(\theta) = \E U^2(\theta)$$

А функции $\fisherinfo_j$
$$\fisherinfo_j(\theta)
= \D_\theta \frac{\partial\ln f(X_j;\theta)}{\partial\theta}
= \int_\RR \left(\frac{\partial\ln f(t;\theta)}{\partial\theta}\right)^2 f(t;\theta) \dd t $$
количеством информации о параметре $\theta$,
содержащейся в $j$-м наблюдении

$$\fisherinfo(\theta) = \sum_{j=1}^n \fisherinfo_j(\theta) = n\fisherinfo_1(\theta)$$
\end{dfn}

\begin{thm}{Неравенство Крамера-Рао}
Для любой несмещённой оценки $T$ параметра $\theta$ справедливо
$$\D_\theta T \geq \frac{1}{\fisherinfo(\theta)}$$
\end{thm}
\begin{proof}
$T=\tau\circ X$ --- несмещённая оценка. Значит
$$\E_\theta T = \int_\samplespace \tau(x)f(x;\theta) \dx = \theta
\qquad \left|\frac\partial{\partial\theta}\right.$$
$$\begin{aligned}
1
&=\frac\partial{\partial\theta}\int_\samplespace \tau(x)f(x;\theta)\dx
 = \int_\samplespace \tau(x) \frac\partial{\partial\theta} f(x;\theta)\dx
 = \int_\samplespace \tau(x) \frac{\partial\ln f}{\partial\theta}(x;\theta)f(x;\theta)\dx \\
&= \E_\theta(T U) = \E_\theta((T - \theta)(U-0)) + \underbrace{\theta\E_\theta U}_{=0}
 = \cov(T,U)
 \leq \sqrt{\D T \D U}
\end{aligned}$$
$$\D T \geq \frac{1}{\D U} = \frac{1}{\fisherinfo(\theta)}$$
\end{proof}
