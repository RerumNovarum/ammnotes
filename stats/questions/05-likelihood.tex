\subsection{Функция правдоподобия. Неравенство Крамера-Рао}

Пусть $f_\theta:\RR\to\ab01$ --- плотность распределения r.v. $\xi$
при данном значении параметра $\theta$,
$X = (X_1, \dotsc, X_n)$ --- выборка из $\pop(\xi)$,
$x = X(\omega)$ --- реализация выборки.
Символом $f_\theta:\samplespace\to\ab01$
будем обозначать плотность распределения выборки
$f_\theta(x) = \prod_{j=1}^n f_\theta(x_j)$

\begin{dfn}{Функция правдоподобия (likelihood fuction)}
При фиксированном $x\in\samplespace$
функция $L_x: \theta\mapsto f_\theta(x)$ называется функцией правдоподобия.

Далее будем считать,
что при любом $x$
отображение $\theta\mapsto f_\theta(x)$ дифференцируемо
\end{dfn}

\begin{dfn}{Вклад (score) выборки}
При каждом $\theta\in\Theta$
введём случайную величину $L_X(\theta)$,
реализация $L_x(\theta)$ которой
есть правдоподобие значения параметра $\theta$ при данной реализации выборки $x$.
Обозначим
 ${L_{X_j}: \theta\mapsto (f_\theta\circ X_j)}$,
 ${L_{x_j}: \theta\mapsto f_\theta(x_j)}$,

Случайная величина $U$
$$U = \frac{\partial \ln L_X(\theta)}{\partial\theta}$$
$$U(\omega)
= u(x) = \sum_{j=1}^n{\frac{\partial \ln L_{x_j}(\theta)}{\partial\theta}}$$
Называется вкладом выборки $X$
\end{dfn}

\begin{dfn}{Регулярная статистическая модель}
Статистическая модель,
позволяющая
дифференцировать (всякие $\int L$ и вообще всё что вздумается) по $\theta$,
переставлять операторы интегрирования и дифференцирования,
и разрешающая прочий матан называется регулярной

Далее рассматриваются регулярные модели
\end{dfn}

\begin{thm}{Свойства функции правдоподобия и вклада}
$$\E_\theta L_X(\theta) = \int_\samplespace f_\theta(x) \dx = 1 \quad\forall\theta\in\Theta$$
$$\E_\theta U = 0$$
\end{thm}
\begin{proof}
Первое равенство естественно.
Продифференцируем его:
$$0 = \frac{\partial\int_\samplespace L_x(\theta) \dx}{\partial\theta}
= \int_\samplespace \frac{\partial L_x(\theta)}{\partial\theta} \dx$$
Заметим
$$U(\omega) = \frac{\partial\ln L_x(\theta)}{\partial} L_x(\theta)
= \frac{\partial L_x(\theta)}{\partial\theta)}\frac{1}{L_x(\theta)} L_x(\theta)
= \frac{\partial L_x(\theta)}{\partial\theta)}$$
Значит
$$0
= \int_\samplespace \frac{\partial\ln L_x(\theta)}{\partial} L_x(\theta) \dx
= \int_\samplespace u(x) f_\theta(x) \dx
= \E_\theta U$$
\end{proof}

