\subsection{Эмпирическая функция распределения, гистограмма}

Пусть $A\subset\Omega_0$ событие,
происходящее в ходе испытания с вероятностью $\pr A = p$,
и пусть эксперимент состоит в проведении $n$ таких независимых испытаний

Тогда
$$\Omega = \prod_{j=1}^n \Omega_0$$
А случайная величина
$$X_j =
I_\{\omega ; \omega_j \in A\} =
\left\{
\begin{aligned}
1; && \omega_j\in A \\
0; && \omega_j\notin A
\end{aligned}\right.$$
является индикатором того, что в ходе $j$-го испытания случилось событие~$A$

Пусть r.v. $k = \sum_{j=1}^n X_j$ --- число проявлений $A$ в ходе эксперимента. \\
Введём r.v. $p_n^* = \frac{1}{n} \sum_{j=1}^n X_j$. \\
Очевидно $\E p_n^* = p$.\\
Кроме того, из ЗБЧ в форме Бернулли следует
$\lim_{n\to\infty} \pr\{ |p_n^* - p| < \varepsilon \} = 1 \quad \forall\varepsilon>0$

Таким образом, значение случайной величины $p_n^*$
можно считать приближённой оценкой величины $p$


Пусть теперь
$X = (X_1, \dotsc, X_n)$ --- выборка объёма $n$ из генеральной совокупности $\pop(\xi)$,
$x = (x_1, \dotsc, x_n)$ --- реализация.

\begin{dfn}{Порядковые статистики}
Каждой реализации $x$ можно сопоставить в соответствие его перестановку
$x_{(1)} \leq \dotsb \leq x_{(n)}$,

$j$-й порядковой статистикой назвается
случайная величина $X_{(j)}$,
при каждой реализации $X(\omega)=x$,
принимает значение $X_{(j)}(\omega) = x_{(j)}$
\end{dfn}

\begin{dfn}{Вариационный ряд}
Случайный вектор $(X_{(1)}, \dotsc, X_{(n)})$ называется вариационным рядом
\end{dfn}

\begin{dfn}{Эмпирическая функция распределения}
Для каждого $t\in\RR$ зададим случайную величину $\mu_n(x)$,
равную количеству элементов выборки $X$,
значения которых не превосходят $t$:
$$\mu_n(x) = \sum I_{ \{ X_j \leq t \} }$$
Эмпирической функцией распределения, построенной по выборке $X$,
называют случайную функцию $t\mapsto F_n(t)$
$$F_n(x) = \frac{1}{n} \mu_n(t)$$
Её значение в точке $t$ является случайной величиной,
сходящейся по вероятности к значению $F(t)$
теоретической функции распределения

\begin{mbox}
EDF можно перезаписать с помощью функции Хевисайда (Heaviside):
\nopagebreak
$$H(t) =
\left\{\begin{aligned}
0; && t < 0 \\
1; && t \geq 0
\end{aligned}\right.$$
$$F_n(t) = \frac{1}{n} \sum_{j=1}^n H(t - X_{(k)})$$
\end{mbox}
\end{dfn}

\begin{dfn}{Гистограмма}
Разобьём область значений r.v. $\xi$ на равные интервалы $\Delta_i$,
и для каждого $\Delta_i$ подсчитаем число $n_i$ элементов $x_j$ вектора $x$,
попавших в $\Delta_i$, $n = \sum n_i$.

Построим график ступенчатой функции $$t\mapsto \frac{n_i}{n h_i},\quad t\in\Delta_i, h_i=|\Delta_i|$$
Полученный график (при желании, само отображение) называется Гистограммой,
построенной по данной реализации выборки

Соединим середины смежных отрезков этого графика.
Полученная ломанная называется полигоном частот

С уменьшением $\max\{h_i\}$,
гистограмма и полигон частот всё более точно приближают
вероятности попадания в каждый из интервалов разбиения
\end{dfn}
