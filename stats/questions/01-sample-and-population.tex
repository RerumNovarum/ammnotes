\subsection{Случайная выборка, генеральная совокупность, функция распределения выборки}

\begin{dfn}{Выборка (sample)}
Пусть эксперемент состоит в проведении $n$ испытаний,
результат $j$-го из которых
является случайной величиной $X_j: \Omega_j\to\samplespace_j$.

Кортёж из этих случайных величин (случайный вектор)
$X = (X_1, \dotsc, X_n)$ называется (случайной) выборкой,
а r.v. $X_j$ называются элементами выборки

А значение $x = (x_1, \dotsc, x_n) = X(\omega)$ называется реализацией выборки

Далее всегда, если не указано иное,
случайные величины будут обозначаться заглавными буквами,
а их реализации соответствующими строчными

Далее $X_j$ полагаются независимыми
\end{dfn}

\begin{dfn}{Выборочное пространство (sample space)}
Выборочным пространством
называется измеримое пространство
$(\samplespace, \mathcal A)$,
где $\samplespace = \{X(\omega) ; \omega\in\Omega\}$
есть множество возможных значений выборки,
а $\mathcal A$ --- $\sigma$-алгебра в $\samplespace$
\end{dfn}

Особенно важен случай, когда случайные величины $X_j$
являются независимыми и
имеют распределение одной случайной величины $\xi$.
Этот случай соответствует повторению $n$ раз одного эксперемента,
описываемого случайной величиной $\xi$

\begin{dfn}{Генеральная совокупность (population)}
Генеральной совокупностью
называют распределение $\pop(\xi)$ случайной величины $\xi$

Оно может быть задано, например, множеством возможных значений r.v. $\xi$
и её функцией распределения

При этом $X$ называют выборкой из (генеральной совокупности) $\pop(\xi)$
\end{dfn}

\begin{dfn}{Функция распределения выборки}
 $X\in\pop(\xi)$
$$F_X(x) = \pr\{X\leq x\} = \prod\pr\{X_j\leq x_j\} = \prod F_{X_j}(x_j)$$
\end{dfn}
