\subsection{Метод моментов}

$X = (X_1, \dotsc, X_n)$ --- выборка из $\pop(\xi)\in\{F_\theta; \theta\in\Theta\}$.
$\Theta\subset\RR^r$.

Пусть существуют и конечные первые $r$ моментов $\alpha_k(\theta) = \E \xi_\theta^k$.
Рассмотрим выборочнные моменты $A_k = \frac{1}{n}\sum_{j=1}^n X_j^k$.

Метод оценивания состоит
в решении относительно $\theta$ системы равенств
$$a_k(\theta) = A_k(\omega) \qquad k=\overline{1,r}$$

\begin{thm}
Пусть $\tilde\theta$ --- оценка, полученная методом моментов.
${\tilde\theta(\omega) = \phi(A_1(\omega), \dotsc, A_r(\omega))}$.
Если $\phi$ непрерывна и взаимно-однозначна, то
так как ${\lim_{n\to\infty} \pr \{A_k = \alpha_k\} = 1}$,
то и ${\tilde\theta \overset{\pr}{\to} \theta}$
\end{thm}
