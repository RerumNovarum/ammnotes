\subsection{Список вопросов к экзамену по математической
статистике}\label{ux441ux43fux438ux441ux43eux43a-ux432ux43eux43fux440ux43eux441ux43eux432-ux43a-ux44dux43aux437ux430ux43cux435ux43dux443-ux43fux43e-ux43cux430ux442ux435ux43cux430ux442ux438ux447ux435ux441ux43aux43eux439-ux441ux442ux430ux442ux438ux441ux442ux438ux43aux435}

\begin{enumerate}
\def\labelenumi{\arabic{enumi}.}
\item
  \begin{enumerate}
  \def\labelenumii{\arabic{enumii}.}
  \item
    Случайная выборка и генеральная совокупность
  \item
    Функция распределения выборки
  \end{enumerate}
\item
  \begin{enumerate}
  \def\labelenumii{\arabic{enumii}.}
  \item
    Эмпирическая функция распределения
  \item
    Гистограмма
  \end{enumerate}
\item
  Выборочные характеристики. Выборочные моменты
\item
  Точечные оценки и их свойства
\item
  Функция правдоподобия. Неравенство Крамера-Рао
\item
  Метод максимального правдоподобия, свойства оценок максимального
  правдоподобия
\item
  Метод моментов для точечных оценок
\item
  Достаточные статистики
\item
  Интервальные оценки. Доверительные интервалы
\item
  Интервальные оценки.

  Доверительные интервал для дисперсии нормальной генеральной
  совокупности
\item
  Асимптотические свойства оценки максимального правдоподобия.

  Асимптотический доверительный интервал
\item
  Проверка статистических гипотез.

  Критерий Неймана-Пирсона проверки простых гипотез
\item
  Наиболее мощный критерий. Теорема Неймана-Пирсона
\item
  Проверка статистических гипотез о параметрах нормального распределения
\item
  Критерии для сложных гипотез
\item
  Функция мощности при альтернативе
\item
  Критерий согласия \(\chi^2\)-Пирсона
\item
  Критерий согласия Колмогорова
\item
  Критерий однородности Колмогорова-Смирнова
\item
  Критерий однородности \(\chi^2\)
\end{enumerate}
