\subsection{Точечные оценки}

Пусть некоторый процесс
описывается вероятностной моделью $(\Omega, \mathcal A, \pr)$,
где $\Omega$ --- пространство элементарных событий,
$\mathcal A \subset 2^\Omega$ --- $\sigma$-алгебра событий,
${\pr: \mathcal A\to \ab{0}{1}}$ --- вероятностная мера,
а проводимый эксперимент соответствует случайной величине $\xi\in\rvspace$,
с функцией распределения $F$.

Рассмотрим задачу определения распределения случайной величины,
в случае когда известно,
что её функция распределения $F$
принадлежит некоторому классу распределений,
зависящих от параметра
$$F\in\mathcal F = \{F_\theta; \theta\in\Theta\}$$
где $\Theta$ --- множество значений параметра $\theta$.
То есть известно,
что распределение определяется некоторым неизвестным истинным значением $\theta_0$ параметра $\theta$,
и задача сводится к его оценке.

Пусть $X = (X_1, \dotsc, X_n)$ --- выборка из $\pop(\xi)$.
Говорят, что пара $(\samplespace, \mathcal F)$ задаёт ``статистическую модель''.

\begin{dfn}{Статистика}
Измеримая функция, определённая на выборочном пространстве.
$$g: \samplespace\to\RR^r$$
называется \emph{статистикой}.
Также статистикой иногда будет называть
случайную величину $g\circ X$ --- композицию $g$ и выборки $X$
$$(g\circ X)(\omega) = g(x)$$
\end{dfn}

\begin{dfn}{Точечная оценка параметра $\theta$}
есть статистика ${T = t\circ X: \Omega\to\Theta}$,
реализацию $T(\omega) = t(x)$ которой принимают за приближённое значение параметра $\theta$
\end{dfn}

\subsubsection{Характеристики оценок}

\begin{dfn}{Несмещённость (unbiasedness)}
Несмещённой называют такую оценку $T = t\circ X$,
что её математическим ожиданием является искомый параметр $\theta$:
$$\E T = \int_\samplespace t(x) f(x;\theta_0) \dx = \theta_0$$
\end{dfn}

\begin{dfn}{Состоятельность (consistency)}
Оценка $T = t\circ X$ называется состоятельной,
если она сходится по вероятности к оцениваемому параметру:

$$\lim_{n\to\infty}
    \pr\{\omega\in\Omega; \left|T(\omega) - \theta_0\right| < \varepsilon \}
= \lim_{n\to\infty}
    \pr\{\left|t(x) - \theta_0\right| < \varepsilon\}
= 1$$
\end{dfn}

\begin{dfn}{Оптимальность (effectiveness)}
Оценка $T_0 = t_0\circ X$ называется \emph{оптимальной в классе} $\mathcal T$ несмещённых оценок,
если среди всех оценок класса $\mathcal T$,
оценка $T_0$ имеет минимальную дисперсию,
то есть для любого $T\in\mathcal T = t\circ X$
$$\D T_0 \leq \D T$$

Оценка называется \emph{оптимальной},
если она оптимальна в классе \emph{всех} несмещённых оценок
\end{dfn}

\begin{thm}{Единственность оптимальной оценки}
Если две несмещённые оценки $T_1 = t_1\circ X, T_2 = \tau_2\circ X$ параметра $\theta$ оптимальны,
то они \emph{равны почти-всюду} $T_1~\overset{\pr}{=}~T_2$:
  $$\pr\{T_1\neq T_2\} = 0$$
\end{thm}
